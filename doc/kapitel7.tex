\chapter{Fazit und Ausblick}
\section{Fazit}
Das Softwareprojekt hatte das Ziel eine Microservice-Architektur auf eine potenzielle 


\section{Einschränkungen}\label{Einschraenkungen}
Hardware wie Industrierechner die später in Produktionsanlagen eingesetzt werden standen nicht zur Verfügung.
Die Installation des Kubernetes-Cluster mit k3s wurde auf virtuelle private Server realisiert (siehe Abschnitt~\ref{konfig}).
Die Implementierung der Microservices konnte deshalb nicht in einem Produktionsumfeld eingesetzt werden.
Der Anwendungsfall der Webanwendung ist deshalb nur bedingt der Realität entsprechend, da Computer-Vision im Bereich der Anlagentechnik genutzt wird und nicht zur Authorisation von Personal.
Jedoch sind viele der Schritte, ähnlich ausführbar wie auf einem Kubernetes-Cluster mit on-Premise Geräten anstatt von cloudbasierter Hardware.
Der blaupausenartige Aufbau der Entwicklungsschritte gilt auch für den Aufbau von Microservices im Produktionsumfeld.

\section{Ausblick}
