\chapter{Fazit und Ausblick}
\section{Fazit}
Das Ziel war die Konzeption und Implementierung einer Microservice-Architektur auf einem hybrides-Kubernetes-Cluster.
Diese soll als Blaupause für die Vorgehensweise der Implementierung dienen.
Anforderungen wie automatischer Tests der Dienste konnte wegen Zeitmangel nicht mehr realisiert werden.
Die Gesichtserkennung benutzt auch keine Deep-Learning-Algorithmen, sondern Vektorbasierte Algorithmen.
Trotzdem sind die Grundvoraussetzungen für die Auslieferung und Bereitstellung einer solchen Anwendung basierend auf Deep-Learning-Methoden gegeben.
Die Installation und Organisation von containerisierten Arbeitslasten durch die Rancher-Plattform erleichterte die Überwachung der Anwendungen.
Durch die Benutzeroberfläche wurde auch die Komplextität zur Verwaltung des Kubernetes-Clusters reduziert.




\section{Einschränkungen}\label{Einschraenkungen}
Hardware wie Industrierechner die später in Produktionsanlagen eingesetzt werden standen nicht zur Verfügung.
Die Installation des Kubernetes-Cluster mit k3s wurde auf virtuelle private Server realisiert (siehe Abschnitt~\ref{konfig}).
Die Implementierung der Microservices konnte deshalb nicht in einem Produktionsumfeld eingesetzt werden.
Der Anwendungsfall der Webanwendung ist deshalb nur bedingt der Realität entsprechend, da Computer-Vision im Bereich der Anlagentechnik genutzt wird und nicht zur Authorisation von Personal.
Jedoch sind viele der Schritte, ähnlich ausführbar wie auf einem Kubernetes-Cluster mit on-Premise Geräten anstatt von cloudbasierter Hardware.
Der blaupausenartige Aufbau der Entwicklungsschritte gilt auch für den Aufbau von Microservices im Produktionsumfeld.

\section{Ausblick}
Die weitere Umsetzung von Anwendungen im Microservice-Architektur-Stil wird weiter untersucht.
Allem vorran der Flexibilität des Kubernetes-Clusters.