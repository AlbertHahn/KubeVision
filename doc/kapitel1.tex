\chapter{Einleitung}
Die Krones AG bietet Anlagen sowohl für die Getränkeindustrie als auch 
Nahrungsmittelhersteller, von der Prozesstechnik bis hin zur IT-Lösung. 
Die Komplettlinie beeinhaltet auch das Bereitstellen von Software auf den einzelnen Produktionsanlagen. 
Hierfür werden eine Vielzahl von Produktionslinienanwendungen auf den Anlagen installiert, gewartet
und verwaltet. Dementsprechend hoch ist der Aufwand, der Fehleranfälligkeiten sowie fehlende Frameworks, Bibliotheken
und anderer Abhängigkeiten mit sich bring.
Eigene Server müssen für die Kommunikation der Anlagen verbaut und gewartet werden,
was zusätzlich Ressourcen beansprucht und automatisch die Kosten für die Inbetriebnahme einer solchen
Linie erhöhen. Die Weiterentwicklung der zukünftigen Bereitstellung von Produktionsanlagensoftware
erfolgt mithilfe eines \ac*{poc}, welcher die Möglichkeiten einer wartungsfreien Infrastruktur
durch ein continuous Delivery System evaluiert. Dies verläuft in Zusammenarbeit mit dem
Kooperationspartner und Softwarteunternehmen SUSE GmbH, welches das wartungsfreie Betriebssystem
SUSE Linux Enterprise Micro und die multi-cluster Orchestrierungsplattform Rancher anbietet.

Als Grundlage hierfür dient das Open-Source-System Kubernetes, welches zur Automatisierung, Skalierung
und Verwaltung von containerisierten Anwendungen verwendet wird. Künftige Produktionsanlagen sollen mittels zusätzlicher Edge Devices
als Knotenpunkte in einem Kubernetes-Cluster fungieren, Ressourcen teilen, untereinander kommunizieren und Softwarepakete unkompliziert bereitstellen.
Die Integration der kompakten Linux-Rechner ermöglichen den variablen Einsatz von Hardwareressourcen des Kunden, der je nach Leistungsanspruch Knotenpunkte erweitern kann.
Dabei soll es für die einzelnen Anwendungen möglich sein, sowohl auf cloudbasierten als auch auf on-premise Hardware zur Verfügung gestellt zu werden.
Ein hybrides Kubernetes-Cluster ermöglicht es somit, lokale Rechenleistung oder öffentliche Cloudressourcen in der selben Softwareumgebung zu nutzen.
\section{Motivation}
Die Vorteile von Kubernetes und dem stetigen Paradigmenwechsel der Softwarelandschaft im Cloudbereich, welcher
den Wechsel von monolithischen Architekturen zu flexibleren Microservice-Architekturen
bevorzugt, sind das Hauptmotiv der Auswertung neuer, agiler Distributionsmöglichkeiten.
Die Containerisierung von Anwendungen ermöglicht erst die Aufteilung großer Projekte
in kleine unabängige Services, die mittels Orchestrierungsplattformen adäquat konzentiert werden können.
Namhafte Unternehmen wie Netflix, Amazon und Uber entwickeln und verwenden
bereits robuste und komplexe Microservices die containerisiert auf Kubernetes-Plattformen
verwaltet werden \cite{microservice}. 
%Absatz nochmal durchgehen

Durch die Flexibiltät einer solchen Infrastuktur ist es möglich Anwendungsfälle im Bereich
der künstlichen Intelligenz
für die Industrie zu testen. Die Anlage Linatronic AI der Krones AG nutzt bereits Deep-Learning-Technologie,
um in der Linie mittels Vollinspektion Schäden, Dichtflächen oder Seitenwanddicken
zu erkennen und Prozesse zu optimieren \cite{linatronic}. Allgemein sind Anwendungen mit künstlicher Intelligenz durch ihre Komplexität
und Vielzahl an Abhängigkeiten schwierig zu entwickeln und bereitzustellen. 
Eine passende Plattform für Anwendungsfälle mit Bezug zur künstlichen Intelligenz
muss eine Vielzahl an Services anbieten. Zu diesen gehören die Verwaltung von Ressourcen wie Speicher,
Rechenleistung und Verbindungsgeschwindigkeit,
für die Datenübertragung bei der Ausführung einzelner Phasen von der Informationsverarbeitung bis hin zur Evaluierung und Entwicklung von Modellen
im Bereich der künstliche Intelligenz.
\cite{mlops}. 

\section{Zielsetzung}
Ziel dieser Arbeit ist die Entwicklung einer Microservice-Architektur in einem hybriden Kubernetes-Cluster. 
Das Endresultat soll eine Anwendung werden, die mittels einer Weboberfläche, welche über eine Domain erreichbar ist,
ein Login-Verfahren mittels einem Backend-Service ermöglichen der ein Authentifierzungsverfahren
per Gesichtserkennung verwendet. 
Diese Daten sollen schließlich verarbeitet und persistent 
gespeichert werden, um bei erneuten Aufruf der Website bestehen zu bleiben.
Die Konzeption der Anwendung findet containerisiert auf mehreren Software und Hardwareschichten
statt. 
Das ganze System wird auf einem Kubernetes-Cluster bereitgestellt und verwaltet.
Das Bereitstellen eines Services kann bei Vorkonfiguration auf on-premise oder cloudbasierten Ressourcen stattfinden.
Ein Ingress-Controller dient dabei als Loadbalancer und verteilt die Last beim 
Aufrufen der Website und der Kommunikation zwischen den Backend-Services.

