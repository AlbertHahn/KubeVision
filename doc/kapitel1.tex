\chapter{Einleitung}
\section{Motivation}
Die Krones AG bietet Anlagen für die Getränkeindustrie als auch 
Nahrungsmittelhersteller, von der Prozesstechnik bis hin zur IT-Lösung. 
Die Komplettlinie beeinhaltet auch das bereitstellen von Software auf den einzelnen Produktionsanlagen. 
Hierfür werden eine Vielzahl von Produktionslinienanwendungen auf den Anlagen installiert und verwaltet.
Ein riesiger Aufwand der Fehleranfälligkeiten wie fehlende Treiber, Bibliotheken
und anderer Abhängigkeiten mit sich bringt.
Eigene Server müssen für die Kommunikation der Anlagen verbaut und gewartet werden,
was zusätzlich Ressourcen benötigt und automatisch die Kosten für die Inbetriebnahme einer solchen
Linien erhöhen.
Deshalb wird für die Weiterentwicklung der zukünftigen Bereitstellung von Produktionsanlagensoftware
mithilfe eines Proof of Concept (PoC), die Möglichkeiten einer wartungsfreien Infrastruktur
durch ein continuous delivery System evaluiert. Dies verläuft in Zusammenarbeit mit dem
Kooperationspartner und Softwarteunternehmen SUSE GmbH, welches das wartungsfreie Betriebssystem
SUSE Linux Enterprise Micro und die multi-cluster Orchestrierungsplattform Rancher anbietet.

Als Grundlage hierfür wird das Open-Source-System Kubernetes untersucht, welches zur Automatisierung, Skalierung
und Verwaltung von containerisierten Anwendungen dient. Künftige Produktionsanlagen sollen mittels zusätzlichen Edge Devices
als Knotenpunkte in einem Kubernetes Cluster fungieren, Ressourcen teilen, untereinander kommunizieren und Softwarepakete unkompliziert bereitstellen.
Die Integration der kleinen Linux Rechner ermöglichen den Variablen Einsatz von Hardwareressourcen beim Kunden, der je nach Anspruch Knotenpunkte erweitern kann.
Dabei soll es für die einzelnen Anwendungen möglich sein, sowohl auf cloudbasierten als auch auf on-premise Hardware zur Verfügung gestellt zu werden.
Ein hybrides Kubernetes Cluster ermöglicht es somit lokale Rechenleistung oder öffentliche Cloudressourcen in der selben Softwareumgebung zu nutzen.

Die Vorteile von Kubernetes und dem stetigen Paradigmenwechsel der Softwarelandschaft im Cloudbereich, welcher
den Wechsel von monolithischen Architektur zu einer mehr flexibleren microservice Architektur
bevorzugt, sind das Hauptmotiv der Auswertung neuer agiler Distributionsmöglichkeiten.
Namenhafte Unternehmen wie Netflix, Amazon und Uber entwickeln und verwenden
bereits robuste und komplexe microservices die containerisiert auf Plattformen
verwaltet werden \cite{microservice}. 
%Absatz nochmal durchgehen

Durch die Flexibiltät einer solchen infrastuktur ist es möglich Anwendungsfälle
für die industrie zu testen. Die Anlage Linatronic AI der Krones AG nutzt Deep-Learning-Technologie
in der Linie mittels Vollinspektion Schäden, Dichtflächen oder Seitenwanddicken
erkennt, um Prozesse zu optimieren. Allgemein sind Machine learning Anwendungen durch ihre Komplexität
und vielzahl an Abhängigkeiten schwierig zu entwickeln und bereitzustellen. 
Eine passende Plattform für Anwendungsfälle mit Bezug zur künstlichen Intelligenz
muss eine Vielzahl an Services anbieten. Verwaltung von Ressourcen wie Speicher,
Rechenleistung und Verbindungsgeschwindigkeit für die Datenübertragung \cite{mlops}. 

\section{Zielsetzung}
Ziel dieser Arbeit ist die Entwicklung einer Microservice
Architektur in einem hybriden Kubernetes Cluster. Das Endresultat
soll eine Anwendung werden die mittels Weboberfläche ein Login-Verfahren
über einen backend Service ermöglichen der als Authentifierzungsverfahren
Gesichtserkennung verwendet, diese Daten sollen verarbeitet und persistent 
gespeichert werden, um bei erneuten Aufruf der Website bestehen zu bleiben.
Das ganze System wird auf einem Kubernetes Cluster bereitgestellt und verwaltet.



