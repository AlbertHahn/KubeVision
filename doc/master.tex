% LaTeX-Vorlage zur Erstellung einer Abschlussarbeit in der Fakultät Elektrotechnik, Medien und Informatik an der OTH Amberg-Weiden
% Diese Vorlage entstand im Rahmen des Kurses "LaTeX fürs Studium"
% Aktuelle Version: v0.02
% Stand: 06.08.2015
%
% Changelog:
%
% v0.02: 06.08.2015, Anpassung der Vorlage:
% + Persönliche Informationen (Vorname, Name, Titel usw.) werden direkt in die PDF-Dokumenteinstellungen übernommen
% + Korrektur der Verlinkung von Abbildungs- und Tabellenverzeichnis aus dem Inhaltsverzeichnis (phantomsection) bzw. deren Seitenzahl
%   Besten Dank für diesen Hinweis an Jan-Olaf Becker
% + Anpassung des Namens der Fakultät nach deren Umbenennung
%
% v0.01: 14.03.2012, Erstellung der Vorlage

\documentclass[12pt,oneside]{report}
\usepackage[T1]{fontenc}		% Einstellungen fuer Umlaute usw.
\usepackage[utf8x]{inputenc}
\usepackage[ngerman]{babel}



\usepackage{parskip}			% Einstellungen fuer Absaetze: Abstand statt Einrueckung

\usepackage[a4paper,			% Papierformat A4
	    left=2.5cm,				% linker Rand
	    right=2.5cm,			% rechter Rand
	    top=1.5cm,				% oberer Rand
	    bottom=1.5cm,			% unter Rand
	    marginparsep=5mm,		% Abstand der Randnotizen
	    marginparwidth=10mm, 	% Breite der Randnotizen
	    headheight=7mm,			% Hoehe der Kopfzeile
	    headsep=1.2cm,			% Abstand der Kopfzeile
	    footskip=1.5cm,			% Abstand der Fusszeile
	    includeheadfoot]{geometry}

\usepackage{fancyhdr}						% Konfiguration von Kopf- und Fusszeilen
\pagestyle{fancy}							% Seitenstil 'fancy'
\fancyhf{}									% vorhandene Einstellungen loeschen
\setlength{\headwidth}{\textwidth}			% Kopf- und Fusszeile so breit wie der Haupttext
\fancyfoot[R]{\thepage} 					% Festlegung des Seitenstils: Seitenzahlen in der Fusszeile rechts
\fancyfoot[L]{\leftmark}					% Kapitelnr. und -Bezeichnung in der Fusszeile links
\fancyhead[R]{\IhreArbeit}					% "Bachelorarbeit" in der Kopfzeile rechts
\fancyhead[L]{\IhrVorname\ \IhrNachname}	% Vorname und Name in der Kopfzeile links
\renewcommand{\chaptermark}[1]{			% Definition der Ausgabe des Kapitels
  \markboth{Kapitel \thechapter. #1}{}}
\renewcommand{\headrulewidth}{0.5pt}		% Trennlinie zwischen Kopfzeile und Haupttext
\renewcommand{\footrulewidth}{0.5pt}		% Trennlinie zwischen Haupttext und Fusszeile
\fancypagestyle{plain}{					% Anpassung des Seitenstils 'plain' bei Beginn neuer Kapitel
  \fancyhf{}								% Vorbelegung loeschen
  \fancyfoot[C]{\thepage}					% Seitenzeilen in der Fusszeile mittig
  \fancyhead[R]{\IhreArbeit}				% "Bachelorarbeit" in der Kopfzeile rechts
  \fancyhead[L]{\IhrVorname\ \IhrNachname}	% Vorname und Name in der Kopfzeile links
}

\usepackage{amsmath}			% Pakete fuer den Mathematikmodus
\usepackage{amssymb}
\usepackage[intlimits]{empheq}

\usepackage[sc]{mathpazo}		% Schriftart Palatino fuer Haupttext und Mathematikmodus
\usepackage{pifont}				% zusaetzliche Symbole

\usepackage[format=hang,		% Einstellung fuer Bildunterschriften
            font={footnotesize},
            labelfont={bf},
            margin=1cm,
            aboveskip=5pt,
            position=bottom]{caption}

\usepackage{graphicx}							% Einbinden von Graphiken
\usepackage[svgnames,table,hyperref]{xcolor} 	% Verwendung von Farben
\usepackage{svg}
\usepackage{amsmath}
\usepackage{tikz}								% Erstellen von Grafiken
\usetikzlibrary{positioning,arrows,plotmarks} % TikZ-Bibliotheken
%\usepackage{pgfplots}                           % Darstellung von Plots, Funktionen, Graphen usw.



%
% Weitere Pakete
%
%YAML https://www.latex4technics.com/?note=187E
\usepackage{listings}
\usepackage{color}
\renewcommand{\lstlistingname}{Quellcode}

\definecolor{vscodeblue}{RGB}{81,147,202}
\definecolor{vscodered}{RGB}{194,137,113}

\newcommand\YAMLcolonstyle{\color{vscodeblue}\mdseries}
\newcommand\YAMLkeystyle{\color{vscodeblue}\bfseries}
\newcommand\YAMLvaluestyle{\color{vscodered}\mdseries}
\newcommand\YAMLframestyle{\color{black}\mdseries}

\makeatletter

% here is a macro expanding to the name of the language
% (handy if you decide to change it further down the road)
\newcommand\language@yaml{yaml}

\expandafter\expandafter\expandafter\lstdefinelanguage
\expandafter{\language@yaml}
{
  frame = single,
  numbers=left,
  stepnumber=1,
  rulecolor=\YAMLframestyle,
  keywords={true,false,null,y,n},
  keywordstyle=\color{darkgray}\bfseries,
  basicstyle=\small\YAMLkeystyle,                                 % assuming a key comes first
  sensitive=false,
  comment=[l]{\#},
  morecomment=[s]{/*}{*/},
  commentstyle=\color{purple}\ttfamily,
  stringstyle=\YAMLvaluestyle\ttfamily,
  moredelim=[l][\color{orange}]{\&},
  moredelim=[l][\color{magenta}]{*},
  moredelim=**[il][\YAMLcolonstyle{:}\YAMLvaluestyle]{:},   % switch to value style at :
  morestring=[b]',
  morestring=[b]",
  literate =    {---}{{\ProcessThreeDashes}}3
                {>}{{\textcolor{red}\textgreater}}1     
                {|}{{\textcolor{red}\textbar}}1 
                {\ -\ }{{\mdseries\ -\ }}3,
}

% switch to key style at EOL
\lst@AddToHook{EveryLine}{\ifx\lst@language\language@yaml\YAMLkeystyle\fi}
\makeatother

\newcommand\ProcessThreeDashes{\llap{\color{cyan}\mdseries-{-}-}}
%YAML---------

%\lstset{language=Python, basicstyle=\ttfamily, numbers=none}


\definecolor{dkgreen}{rgb}{0,0.6,0}
\definecolor{gray}{rgb}{0.5,0.5,0.5}
\definecolor{mauve}{rgb}{0.58,0,0.82}

\lstset{frame=tb,
  language=python,
  aboveskip=3mm,
  belowskip=3mm,
  showstringspaces=false,
  columns=flexible,
  basicstyle={\ttfamily},
  numbers=none,
  numberstyle=\tiny\color{gray},
  keywordstyle=\color{blue},
  commentstyle=\color{dkgreen},
  stringstyle=\color{mauve},
  breaklines=true,
  breakatwhitespace=true,
  tabsize=3
}


%
%\usepackage[european, siunitx]{circuitikz}	% Darstellung von Schaltungen
%
%\usepackage{enumerate}			% Formatierung nummerierter Listen

\usepackage{microtype,relsize}					% Wird verwendet, um Nachnamen auf Titelseite gesperrt darzustellen
\newcommand*{\Sperren}[1]{\textls*[100]{#1}}

% 
% Persoenliche Angaben
% 
\newcommand*{\IhrVorname}{Albert}
\newcommand*{\IhrNachname}{Hahn}
\newcommand*{\IhrStudiengang}{Medieninformatik}
\newcommand*{\IhreArbeit}{Bachelorarbeit}
\newcommand*{\IhrTitelDE}{Konzeption und Implementierung einer Microservice Architektur in
einem hybriden kubernetes Cluster für industrielle KI-Anwendungsfälle}
\newcommand*{\IhrTitelEN}{Conceptual Design and Implementation of a Microservice Architecture
in a Hybrid Kubernetes Cluster for Industrial AI Use Cases}
\newcommand*{\IhrBearbeitungszeitraumVON}{4. Oktober 2021}
\newcommand*{\IhrBearbeitungszeitraumBIS}{3. März 2022}
\newcommand*{\IhrErstpruefer}{Prof. Dr.-Ing. Christoph Neumann}
\newcommand*{\IhrZweitpruefer}{Prof. Dr. Dieter Meiller}
\newcommand*{\IhreFirma}{Krones AG, Neutraubling}
\newcommand*{\IhrFirmenbetreuer}{Ottmar Amann}
\newcommand*{\IhreZusammenfassung}{%
Das Ziel dieser Bachelorarbeit ist es, eine flexible
und nahtlose Lösung für ein Hybrides Cluster aus on-premise Edge Devices und Cloud
Ressourcen bereitzustellen. Produktionslinienanwendungen/Microservices sollen
zukünftig beliebig skalierbar und agil sein, dabei sollen für die Anwendungen generell
keine Differenzierung zwischen offline und online Ressource getroffen werden. Im Zuge
dessen wird die Umsetzbarkeit und Relevanz von cloudbasierten Microservices im
Bereich der künstlichen Intelligenz auf einer zukünftigen Produktionsanlage untersucht.
}
\newcommand*{\IhreSchluesselwoerter}{}
\newcommand{\secref}[1]{\autoref{#1}. \nameref{#1}}

\usepackage[bookmarks, raiselinks, pageanchor, % PDF-Einstellungen
            hyperindex, colorlinks,
            citecolor=black, linkcolor=black,
            urlcolor=black, filecolor=black,
            menucolor=black]{hyperref}
\hypersetup{pdftitle={\IhrTitelDE},%
            pdfauthor={\IhrVorname\ \IhrNachname},%
            pdfsubject={\IhreArbeit},%
            pdfkeywords={\IhreSchluesselwoerter}}

%
% Beginn des Textteils
%
\begin{document}
  \pagenumbering{roman}
  \begin{titlepage}					% Titelseite
    \thispagestyle{empty}
    \begin{center}
      \Large
      Ostbayerische Technische Hochschule Amberg-Weiden\\
      Fakultät Elektrotechnik, Medien und Informatik\\[1cm]
      Studiengang \IhrStudiengang\\[1cm]
      \textbf{\IhreArbeit}\\[1cm]
      von\\[1cm]
      \IhrVorname\ \Sperren{\textbf{\IhrNachname}}\\[1cm]
      \textbf{\IhrTitelDE}\\[1cm]
      \IhrTitelEN
    \end{center}
  \end{titlepage}
  \clearpage
  \thispagestyle{empty}			% 1. Seite soll eine Leerseite sein (dazu muss ein Trick verwendet werden)
  \mbox{}
  \clearpage
  \thispagestyle{empty}			% 2. Seite wie Titelseite, aber mit zusaetzlichen Angaben
  \begin{center}
    \Large
    Ostbayerische Technische Hochschule Amberg-Weiden\\
    Fakultät Elektrotechnik, Medien und Informatik\\[1cm]
    Studiengang \IhrStudiengang\\[1cm]
    \textbf{\IhreArbeit}\\[1cm]
    von\\[1cm]
    \IhrVorname\ \Sperren{\textbf{\IhrNachname}}\\[1cm]
    \textbf{\IhrTitelDE}\\[1cm]
    \IhrTitelEN
  \end{center}
  \vspace*{4cm}
  \begin{tabbing}
    \underbar{Bearbeitungszeitraum:}\qquad\= von\qquad\=\IhrBearbeitungszeitraumVON\\
                                          \> bis      \>\IhrBearbeitungszeitraumBIS
  \end{tabbing}
  \vspace*{1cm}
  \underbar{1. Prüfer:}\qquad\IhrErstpruefer\par 
  \underbar{2. Prüfer:}\qquad\IhrZweitpruefer
  \clearpage
  % formblatt_para12apo.tex
%

\thispagestyle{empty}				% Formblatt Bestaetigung nach Paragraph 12 APO
\begin{minipage}{0.65\textwidth}
  Ostbayerische Technische Hochschule Amberg-Weiden\\
  Fakultät Elektrotechnik, Medien und Informatik\\[1.5cm]
\end{minipage}
\begin{minipage}{0.35\textwidth}
  \raggedleft
\definecolor{ca69788}{RGB}{166,151,136}
\definecolor{cf68712}{RGB}{246,135,18}
\begin{tikzpicture}[y=0.80pt, x=0.8pt,yscale=-0.35,xscale=0.35, inner sep=0pt, outer sep=0pt]
\begin{scope}[cm={{1.25,0.0,0.0,-1.25,(0.0,259.45)}}]
  \begin{scope}[scale=0.100]
    \path[fill=ca69788,nonzero rule] (104.9570,451.7810) .. controls
      (102.2700,460.3200) and (93.2422,473.2500) .. (75.1836,473.2500) .. controls
      (63.7188,473.2500) and (53.7148,467.3910) .. (48.8398,458.3590) .. controls
      (42.9844,447.3910) and (40.2969,432.5000) .. (40.2969,412.0120) .. controls
      (40.2969,382.7300) and (45.1758,364.4300) .. (55.4258,357.1090) .. controls
      (60.7891,353.2110) and (67.6211,351.2500) .. (75.6719,351.2500) .. controls
      (99.3398,351.2500) and (109.3480,369.3090) .. (109.3480,412.5000) .. controls
      (109.3480,429.8200) and (107.8830,442.2620) .. (104.9570,451.7810) --
      cycle(110.8130,332.9490) .. controls (100.5630,327.3400) and
      (91.0430,325.1480) .. (76.8945,325.1480) .. controls (51.2734,325.1480) and
      (34.6875,332.2190) .. (21.7539,349.0590) .. controls (8.8242,365.6480) and
      (2.2383,387.1210) .. (2.2383,412.0120) .. controls (2.2383,448.6090) and
      (16.1445,477.8910) .. (40.5430,491.3010) .. controls (50.5469,496.6720) and
      (62.9883,499.6020) .. (75.6719,499.6020) .. controls (120.8160,499.6020) and
      (148.6290,466.6600) .. (148.6290,413.4690) .. controls (148.6290,375.1720) and
      (135.4530,346.3790) .. (110.8130,332.9490);
    \path[fill=ca69788,nonzero rule] (213.7770,323.9300) .. controls
      (198.4060,323.9300) and (181.5700,328.8090) .. (163.2700,338.3320) --
      (174.9840,362.2380) .. controls (184.9840,356.1290) and (202.3090,348.0820) ..
      (216.4610,348.0820) .. controls (225.7300,348.0820) and (233.0510,354.1800) ..
      (233.0510,362.2380) .. controls (233.0510,370.7810) and (226.9530,375.1720) ..
      (213.7770,377.6090) -- (199.1410,380.2890) .. controls (190.8440,381.7500) and
      (180.5980,387.6090) .. (176.2070,392.9800) .. controls (171.8130,398.3400) and
      (169.1250,407.3710) .. (169.1250,415.4220) .. controls (169.1250,439.8200) and
      (188.4020,456.1720) .. (217.4380,456.1720) .. controls (237.4450,456.1720) and
      (250.6210,450.0700) .. (262.0900,444.4610) -- (251.3520,422.5000) .. controls
      (238.9100,428.8400) and (229.8830,431.5310) .. (220.6090,431.5310) .. controls
      (211.0940,431.5310) and (204.7500,426.6480) .. (204.7500,419.3320) .. controls
      (204.7500,412.9800) and (208.8950,409.5700) .. (220.3670,406.6410) --
      (235.4920,402.7300) .. controls (250.8630,398.8320) and (255.9840,394.1910) ..
      (260.3830,388.5820) .. controls (265.0160,382.7300) and (267.2110,375.6480) ..
      (267.2110,367.3590) .. controls (267.2110,341.4880) and (245.7420,323.9300) ..
      (213.7770,323.9300);
    \path[fill=ca69788,nonzero rule] (329.8950,324.8980) .. controls
      (313.3010,324.8980) and (300.1290,332.2190) .. (296.2270,343.1990) .. controls
      (294.2730,348.5700) and (294.0270,351.0120) .. (294.0270,362.4800) --
      (294.0270,430.3090) -- (281.5820,430.3090) -- (281.5820,452.7500) --
      (294.0270,452.7500) .. controls (294.0270,464.9490) and (294.0270,473.0000) ..
      (295.2500,482.2810) -- (328.4300,490.5700) .. controls (327.2110,479.1090) and
      (326.4800,465.4410) .. (326.4800,452.7500) -- (355.7580,452.7500) --
      (347.4610,430.3090) -- (326.4800,430.3090) -- (326.4800,367.6020) .. controls
      (326.4800,351.7380) and (329.4020,347.5900) .. (340.6290,347.5900) .. controls
      (343.5550,347.5900) and (346.4840,348.3320) .. (352.3400,350.0310) --
      (356.4880,330.5200) .. controls (346.9730,326.6090) and (338.4340,324.8980) ..
      (329.8950,324.8980);
    \path[fill=ca69788,nonzero rule] (447.7380,412.9800) .. controls
      (445.2970,423.7190) and (438.4690,428.8400) .. (429.9260,428.8400) .. controls
      (421.3870,428.8400) and (415.5310,423.4800) .. (411.3870,418.8400) --
      (411.3870,361.2620) .. controls (415.7770,357.3520) and (420.8980,353.2110) ..
      (429.6840,353.2110) .. controls (437.7340,353.2110) and (442.8590,356.3790) ..
      (445.5430,362.9610) .. controls (448.4730,369.8010) and (449.2030,376.3910) ..
      (449.2030,391.7620) .. controls (449.2030,402.9800) and (448.9610,407.6210) ..
      (447.7380,412.9800) -- cycle(463.1090,335.1480) .. controls
      (454.5700,328.3200) and (446.0310,325.3910) .. (435.7810,325.3910) .. controls
      (423.5860,325.3910) and (415.0430,329.0510) .. (407.9690,336.8590) .. controls
      (406.9920,332.4690) and (406.7460,331.2500) .. (404.5510,327.8320) --
      (375.2730,327.8320) .. controls (377.7150,333.4410) and (378.4450,337.1020) ..
      (378.4450,354.4300) -- (378.4450,470.0820) .. controls (378.4450,485.4490) and
      (377.9610,493.7500) .. (376.2500,500.8200) -- (409.6800,508.6290) .. controls
      (410.6520,500.0900) and (410.8980,494.9690) .. (410.8980,486.4300) --
      (410.8980,456.8980) .. controls (410.8980,453.2380) and (410.6520,447.8790) ..
      (410.1680,445.9220) .. controls (416.5080,452.7500) and (425.2930,455.9300) ..
      (436.7580,455.9300) .. controls (467.2580,455.9300) and (486.2890,431.5310) ..
      (486.2890,392.4920) .. controls (486.2890,367.1090) and (478.4800,347.3520) ..
      (463.1090,335.1480);
    \path[fill=ca69788,nonzero rule] (572.3870,382.4920) .. controls
      (549.6910,382.4920) and (541.8870,378.3400) .. (541.8870,363.4490) .. controls
      (541.8870,353.6990) and (547.9880,347.1090) .. (556.2850,347.1090) .. controls
      (562.3830,347.1090) and (568.4800,350.2810) .. (573.3590,355.6410) --
      (573.8480,382.4920) -- (572.3870,382.4920) -- cycle(600.1990,321.4880) ..
      controls (592.6370,324.6600) and (585.8050,330.2700) .. (582.6330,336.6210) ..
      controls (580.1950,334.1720) and (577.5120,331.7300) .. (575.0660,330.0310) ..
      controls (568.9690,325.6290) and (560.1880,323.1990) .. (549.9380,323.1990) ..
      controls (522.1250,323.1990) and (506.9960,337.3520) .. (506.9960,362.2380) ..
      controls (506.9960,391.5120) and (527.2460,405.1800) .. (567.0160,405.1800) ..
      controls (569.4570,405.1800) and (571.6560,405.1800) .. (574.3360,404.9300) --
      (574.3360,410.0510) .. controls (574.3360,423.9610) and (571.6560,428.6020) ..
      (559.6950,428.6020) .. controls (549.2030,428.6020) and (537.0080,423.4800) ..
      (523.5900,414.4490) -- (509.6840,437.8710) .. controls (516.2700,442.0200) and
      (521.1480,444.4610) .. (529.9340,448.1210) .. controls (542.1290,453.2380) and
      (552.6210,455.4410) .. (564.0940,455.4410) .. controls (585.0740,455.4410) and
      (599.4690,447.6290) .. (604.3480,433.7190) .. controls (606.0550,428.6020) and
      (606.7850,424.6990) .. (606.5430,411.2700) -- (605.8130,369.3090) .. controls
      (605.5660,355.6410) and (606.5430,349.7890) .. (617.5230,341.4880) --
      (600.1990,321.4880);
    \path[fill=ca69788,nonzero rule] (702.8910,325.8790) .. controls
      (694.3520,301.7190) and (688.0080,291.2420) .. (678.9800,284.6480) .. controls
      (670.6840,278.5510) and (659.9490,274.6410) .. (648.7270,273.1800) --
      (637.5040,294.6480) .. controls (644.5780,296.6020) and (652.8750,299.5310) ..
      (657.7540,302.9410) .. controls (661.4100,305.6290) and (664.3400,309.0390) ..
      (667.0270,313.1910) .. controls (670.1950,318.3200) and (671.1720,320.5120) ..
      (674.1020,327.8320) -- (665.8050,327.8320) .. controls (661.8980,339.5510) and
      (655.3130,359.0590) .. (654.0940,362.9610) -- (624.5700,450.8010) --
      (657.9960,454.7110) -- (679.7110,381.7500) .. controls (681.6640,374.4300) and
      (685.8130,357.6020) .. (686.3010,355.8910) .. controls (686.3010,356.6210) and
      (688.7380,369.8010) .. (690.2030,375.8980) .. controls (691.1840,380.0390) and
      (693.1290,387.3590) .. (695.0820,393.2190) -- (713.8710,452.7500) --
      (748.2700,452.7500) -- (702.8910,325.8790);
    \path[fill=ca69788,nonzero rule] (832.9100,405.4220) .. controls
      (832.9100,414.6910) and (831.9340,419.5700) .. (829.0080,424.2110) .. controls
      (825.8360,429.0900) and (821.1990,431.5310) .. (814.6130,431.5310) .. controls
      (802.1680,431.5310) and (795.0940,421.7700) .. (795.0940,404.4410) --
      (795.0940,403.9610) -- (832.9100,403.9610) -- (832.9100,405.4220) --
      cycle(794.6050,380.0390) -- (794.6050,379.0700) .. controls
      (794.6050,359.8010) and (804.1210,348.8090) .. (820.9570,348.8090) .. controls
      (832.1800,348.8090) and (842.6720,352.9610) .. (852.6760,361.2620) --
      (865.3590,341.7380) .. controls (850.9690,330.0310) and (835.8400,324.4100) ..
      (818.2700,324.4100) .. controls (782.4060,324.4100) and (759.2300,349.7890) ..
      (759.2300,389.0700) .. controls (759.2300,411.5200) and (763.8630,426.3980) ..
      (774.8440,438.6020) .. controls (785.0900,450.0700) and (797.5350,455.4410) ..
      (814.1250,455.4410) .. controls (828.5200,455.4410) and (842.1840,450.5590) ..
      (850.2340,442.2620) .. controls (861.7030,430.5510) and (866.8240,413.7190) ..
      (866.8240,387.6090) -- (866.8240,380.0390) -- (794.6050,380.0390);
    \path[fill=ca69788,nonzero rule] (955.8910,424.2110) .. controls
      (952.7150,425.9100) and (950.0350,426.6480) .. (946.3710,426.6480) .. controls
      (939.0550,426.6480) and (932.4650,423.2300) .. (926.3670,416.1600) --
      (926.3670,327.8320) -- (893.6720,327.8320) -- (893.6720,411.2700) .. controls
      (893.6720,428.1090) and (891.7190,440.8010) .. (889.0310,447.8790) --
      (918.3160,455.6800) .. controls (921.2380,450.5590) and (922.9490,444.9490) ..
      (923.4380,437.8710) .. controls (930.5120,447.3910) and (940.5200,455.6800) ..
      (952.7150,455.6800) .. controls (957.5980,455.6800) and (959.7890,455.1990) ..
      (964.9180,453.0000) -- (955.8910,424.2110);
    \path[fill=ca69788,nonzero rule] (981.9920,327.8320) -- (981.9920,450.5590) --
      (1014.6900,455.6800) -- (1014.6900,327.8320) -- (981.9920,327.8320) --
      cycle(998.3400,466.8980) .. controls (987.3590,466.8980) and
      (978.3280,475.9300) .. (978.3280,487.1600) .. controls (978.3280,498.3790) and
      (987.6020,507.4100) .. (998.8280,507.4100) .. controls (1009.8000,507.4100)
      and (1018.5900,498.3790) .. (1018.5900,487.1600) .. controls
      (1018.5900,475.9300) and (1009.5600,466.8980) .. (998.3400,466.8980);
    \path[fill=ca69788,nonzero rule] (1089.5900,323.9300) .. controls
      (1074.2200,323.9300) and (1057.3800,328.8090) .. (1039.0800,338.3320) --
      (1050.8000,362.2380) .. controls (1060.8000,356.1290) and (1078.1200,348.0820)
      .. (1092.2700,348.0820) .. controls (1101.5400,348.0820) and
      (1108.8600,354.1800) .. (1108.8600,362.2380) .. controls (1108.8600,370.7810)
      and (1102.7600,375.1720) .. (1089.5900,377.6090) -- (1074.9500,380.2890) ..
      controls (1066.6600,381.7500) and (1056.4100,387.6090) .. (1052.0200,392.9800)
      .. controls (1047.6200,398.3400) and (1044.9400,407.3710) ..
      (1044.9400,415.4220) .. controls (1044.9400,439.8200) and (1064.2100,456.1720)
      .. (1093.2500,456.1720) .. controls (1113.2600,456.1720) and
      (1126.4300,450.0700) .. (1137.9000,444.4610) -- (1127.1600,422.5000) ..
      controls (1114.7200,428.8400) and (1105.6900,431.5310) .. (1096.4200,431.5310)
      .. controls (1086.9000,431.5310) and (1080.5600,426.6480) ..
      (1080.5600,419.3320) .. controls (1080.5600,412.9800) and (1084.7100,409.5700)
      .. (1096.1800,406.6410) -- (1111.3000,402.7300) .. controls
      (1126.6700,398.8320) and (1131.8000,394.1910) .. (1136.1900,388.5820) ..
      controls (1140.8300,382.7300) and (1143.0200,375.6480) .. (1143.0200,367.3590)
      .. controls (1143.0200,341.4880) and (1121.5500,323.9300) ..
      (1089.5900,323.9300);
    \path[fill=ca69788,nonzero rule] (1247.4200,332.7110) .. controls
      (1238.6300,327.5900) and (1228.8700,324.8980) .. (1216.9200,324.8980) ..
      controls (1182.5100,324.8980) and (1162.5100,348.8090) .. (1162.5100,389.3200)
      .. controls (1162.5100,418.1090) and (1173.4900,437.1410) ..
      (1188.1300,447.1410) .. controls (1196.4200,452.7500) and (1208.6200,456.4220)
      .. (1219.1200,456.4220) .. controls (1227.4100,456.4220) and
      (1236.4400,454.4610) .. (1243.2700,450.8010) .. controls (1247.9100,448.3590)
      and (1250.1000,446.6600) .. (1255.4700,442.0200) -- (1239.6100,420.5510) ..
      controls (1233.0200,426.6480) and (1225.9500,430.3090) .. (1219.8400,430.3090)
      .. controls (1205.2100,430.3090) and (1198.6200,417.6210) ..
      (1198.6200,388.3400) .. controls (1198.6200,371.9880) and (1200.8200,362.2380)
      .. (1204.9600,356.8710) .. controls (1208.3800,352.4800) and
      (1213.9900,349.7890) .. (1219.6000,349.7890) .. controls (1227.1600,349.7890)
      and (1234.0000,352.9610) .. (1242.0500,360.0390) -- (1244.0000,361.7500) --
      (1258.8800,341.9800) .. controls (1254.0000,337.1020) and (1251.8100,335.3980)
      .. (1247.4200,332.7110);
    \path[fill=ca69788,nonzero rule] (1350.1100,327.8320) -- (1350.1100,411.7620) ..
      controls (1350.1100,424.2110) and (1346.6900,428.8400) .. (1337.4200,428.8400)
      .. controls (1329.3700,428.8400) and (1318.8800,423.9610) ..
      (1311.5600,417.3790) -- (1311.5600,327.8320) -- (1278.3700,327.8320) --
      (1278.3700,472.2700) .. controls (1278.3700,483.9800) and (1277.4000,495.6990)
      .. (1275.9300,500.8200) -- (1309.3600,508.6290) .. controls
      (1310.8300,501.8010) and (1311.5600,490.0820) .. (1311.5600,478.1290) --
      (1311.5600,453.2380) .. controls (1311.5600,449.3400) and (1311.0700,444.2110)
      .. (1311.0700,442.7500) .. controls (1319.6100,450.8010) and
      (1333.7600,456.1720) .. (1346.4500,456.1720) .. controls (1362.3000,456.1720)
      and (1374.9900,449.3400) .. (1378.9000,438.3590) .. controls
      (1381.3400,431.2810) and (1382.0700,427.1290) .. (1382.0700,415.1800) --
      (1382.0700,327.8320) -- (1350.1100,327.8320);
    \path[fill=ca69788,nonzero rule] (1486.2500,405.4220) .. controls
      (1486.2500,414.6910) and (1485.2700,419.5700) .. (1482.3500,424.2110) ..
      controls (1479.1700,429.0900) and (1474.5400,431.5310) .. (1467.9500,431.5310)
      .. controls (1455.5100,431.5310) and (1448.4300,421.7700) ..
      (1448.4300,404.4410) -- (1448.4300,403.9610) -- (1486.2500,403.9610) --
      (1486.2500,405.4220) -- cycle(1447.9400,380.0390) -- (1447.9400,379.0700) ..
      controls (1447.9400,359.8010) and (1457.4600,348.8090) .. (1474.3000,348.8090)
      .. controls (1485.5200,348.8090) and (1496.0100,352.9610) ..
      (1506.0200,361.2620) -- (1518.7000,341.7380) .. controls (1504.3100,330.0310)
      and (1489.1800,324.4100) .. (1471.6100,324.4100) .. controls
      (1435.7500,324.4100) and (1412.5700,349.7890) .. (1412.5700,389.0700) ..
      controls (1412.5700,411.5200) and (1417.2000,426.3980) .. (1428.1800,438.6020)
      .. controls (1438.4300,450.0700) and (1450.8700,455.4410) ..
      (1467.4700,455.4410) .. controls (1481.8600,455.4410) and (1495.5200,450.5590)
      .. (1503.5700,442.2620) .. controls (1515.0400,430.5510) and
      (1520.1700,413.7190) .. (1520.1700,387.6090) -- (1520.1700,380.0390) --
      (1447.9400,380.0390);
    \path[fill=ca69788,nonzero rule] (1702.6700,469.1020) -- (1662.1700,469.1020) --
      (1662.1700,327.8320) -- (1627.5200,327.8320) -- (1627.5200,469.1020) --
      (1586.0500,469.1020) -- (1586.0500,497.3980) -- (1708.2900,497.3980) --
      (1702.6700,469.1020);
    \path[fill=ca69788,nonzero rule] (1770.9800,405.4220) .. controls
      (1770.9800,414.6910) and (1770.0000,419.5700) .. (1767.0800,424.2110) ..
      controls (1763.9000,429.0900) and (1759.2600,431.5310) .. (1752.6800,431.5310)
      .. controls (1740.2400,431.5310) and (1733.1600,421.7700) ..
      (1733.1600,404.4410) -- (1733.1600,403.9610) -- (1770.9800,403.9610) --
      (1770.9800,405.4220) -- cycle(1732.6700,380.0390) -- (1732.6700,379.0700) ..
      controls (1732.6700,359.8010) and (1742.1900,348.8090) .. (1759.0200,348.8090)
      .. controls (1770.2500,348.8090) and (1780.7400,352.9610) ..
      (1790.7400,361.2620) -- (1803.4300,341.7380) .. controls (1789.0300,330.0310)
      and (1773.9000,324.4100) .. (1756.3400,324.4100) .. controls
      (1720.4700,324.4100) and (1697.2900,349.7890) .. (1697.2900,389.0700) ..
      controls (1697.2900,411.5200) and (1701.9300,426.3980) .. (1712.9100,438.6020)
      .. controls (1723.1600,450.0700) and (1735.6000,455.4410) ..
      (1752.1900,455.4410) .. controls (1766.5900,455.4410) and (1780.2500,450.5590)
      .. (1788.3100,442.2620) .. controls (1799.7700,430.5510) and
      (1804.8900,413.7190) .. (1804.8900,387.6090) -- (1804.8900,380.0390) --
      (1732.6700,380.0390);
    \path[fill=ca69788,nonzero rule] (1909.3300,332.7110) .. controls
      (1900.5500,327.5900) and (1890.7900,324.8980) .. (1878.8300,324.8980) ..
      controls (1844.4300,324.8980) and (1824.4200,348.8090) .. (1824.4200,389.3200)
      .. controls (1824.4200,418.1090) and (1835.4100,437.1410) ..
      (1850.0400,447.1410) .. controls (1858.3300,452.7500) and (1870.5300,456.4220)
      .. (1881.0300,456.4220) .. controls (1889.3200,456.4220) and
      (1898.3500,454.4610) .. (1905.1800,450.8010) .. controls (1909.8200,448.3590)
      and (1912.0200,446.6600) .. (1917.3800,442.0200) -- (1901.5200,420.5510) ..
      controls (1894.9400,426.6480) and (1887.8600,430.3090) .. (1881.7600,430.3090)
      .. controls (1867.1200,430.3090) and (1860.5300,417.6210) ..
      (1860.5300,388.3400) .. controls (1860.5300,371.9880) and (1862.7300,362.2380)
      .. (1866.8800,356.8710) .. controls (1870.2900,352.4800) and
      (1875.9000,349.7890) .. (1881.5200,349.7890) .. controls (1889.0800,349.7890)
      and (1895.9100,352.9610) .. (1903.9600,360.0390) -- (1905.9100,361.7500) --
      (1920.8000,341.9800) .. controls (1915.9200,337.1020) and (1913.7300,335.3980)
      .. (1909.3300,332.7110);
    \path[fill=ca69788,nonzero rule] (2012.0200,327.8320) -- (2012.0200,411.7620) ..
      controls (2012.0200,424.2110) and (2008.6100,428.8400) .. (1999.3300,428.8400)
      .. controls (1991.2800,428.8400) and (1980.7900,423.9610) ..
      (1973.4700,417.3790) -- (1973.4700,327.8320) -- (1940.2900,327.8320) --
      (1940.2900,472.2700) .. controls (1940.2900,483.9800) and (1939.3100,495.6990)
      .. (1937.8500,500.8200) -- (1971.2700,508.6290) .. controls
      (1972.7400,501.8010) and (1973.4700,490.0820) .. (1973.4700,478.1290) --
      (1973.4700,453.2380) .. controls (1973.4700,449.3400) and (1972.9800,444.2110)
      .. (1972.9800,442.7500) .. controls (1981.5200,450.8010) and
      (1995.6700,456.1720) .. (2008.3600,456.1720) .. controls (2024.2200,456.1720)
      and (2036.9100,449.3400) .. (2040.8100,438.3590) .. controls
      (2043.2500,431.2810) and (2043.9800,427.1290) .. (2043.9800,415.1800) --
      (2043.9800,327.8320) -- (2012.0200,327.8320);
    \path[fill=ca69788,nonzero rule] (2148.1600,327.8320) -- (2148.1600,409.0820) ..
      controls (2148.1600,423.2300) and (2145.7300,427.3790) .. (2137.1800,427.3790)
      .. controls (2130.6000,427.3790) and (2122.0600,422.9880) ..
      (2114.5000,416.1600) -- (2114.5000,327.8320) -- (2081.8000,327.8320) --
      (2081.8000,418.3520) .. controls (2081.8000,429.0900) and (2080.3400,439.3400)
      .. (2077.4100,447.6290) -- (2106.4400,455.9300) .. controls
      (2109.3700,450.8010) and (2111.0800,445.4300) .. (2111.0800,440.3090) ..
      controls (2115.9600,443.7300) and (2120.1000,446.6600) .. (2125.4700,449.5820)
      .. controls (2132.0700,453.0000) and (2140.6000,454.9490) ..
      (2147.9300,454.9490) .. controls (2161.8400,454.9490) and (2174.0200,447.6290)
      .. (2177.9300,436.8910) .. controls (2179.6500,432.2620) and
      (2180.3700,426.8910) .. (2180.3700,419.0820) -- (2180.3700,327.8320) --
      (2148.1600,327.8320);
    \path[fill=ca69788,nonzero rule] (2218.2000,327.8320) -- (2218.2000,450.5590) --
      (2250.9000,455.6800) -- (2250.9000,327.8320) -- (2218.2000,327.8320) --
      cycle(2234.5500,466.8980) .. controls (2223.5700,466.8980) and
      (2214.5300,475.9300) .. (2214.5300,487.1600) .. controls (2214.5300,498.3790)
      and (2223.8100,507.4100) .. (2235.0400,507.4100) .. controls
      (2246.0200,507.4100) and (2254.8000,498.3790) .. (2254.8000,487.1600) ..
      controls (2254.8000,475.9300) and (2245.7600,466.8980) ..
      (2234.5500,466.8980);
    \path[fill=ca69788,nonzero rule] (2325.7800,323.9300) .. controls
      (2310.4100,323.9300) and (2293.5700,328.8090) .. (2275.2900,338.3320) --
      (2286.9900,362.2380) .. controls (2296.9900,356.1290) and (2314.3200,348.0820)
      .. (2328.4800,348.0820) .. controls (2337.7500,348.0820) and
      (2345.0600,354.1800) .. (2345.0600,362.2380) .. controls (2345.0600,370.7810)
      and (2338.9600,375.1720) .. (2325.7800,377.6090) -- (2311.1500,380.2890) ..
      controls (2302.8500,381.7500) and (2292.6000,387.6090) .. (2288.2200,392.9800)
      .. controls (2283.8300,398.3400) and (2281.1300,407.3710) ..
      (2281.1300,415.4220) .. controls (2281.1300,439.8200) and (2300.4100,456.1720)
      .. (2329.4500,456.1720) .. controls (2349.4500,456.1720) and
      (2362.6400,450.0700) .. (2374.1000,444.4610) -- (2363.3600,422.5000) ..
      controls (2350.9200,428.8400) and (2341.8900,431.5310) .. (2332.6200,431.5310)
      .. controls (2323.1100,431.5310) and (2316.7600,426.6480) ..
      (2316.7600,419.3320) .. controls (2316.7600,412.9800) and (2320.9200,409.5700)
      .. (2332.3800,406.6410) -- (2347.5000,402.7300) .. controls
      (2362.8700,398.8320) and (2368.0100,394.1910) .. (2372.3800,388.5820) ..
      controls (2377.0300,382.7300) and (2379.2200,375.6480) .. (2379.2200,367.3590)
      .. controls (2379.2200,341.4880) and (2357.7500,323.9300) ..
      (2325.7800,323.9300);
    \path[fill=ca69788,nonzero rule] (2483.6300,332.7110) .. controls
      (2474.8400,327.5900) and (2465.0800,324.8980) .. (2453.1300,324.8980) ..
      controls (2418.7300,324.8980) and (2398.7100,348.8090) .. (2398.7100,389.3200)
      .. controls (2398.7100,418.1090) and (2409.7100,437.1410) ..
      (2424.3400,447.1410) .. controls (2432.6400,452.7500) and (2444.8200,456.4220)
      .. (2455.3300,456.4220) .. controls (2463.6100,456.4220) and
      (2472.6600,454.4610) .. (2479.4700,450.8010) .. controls (2484.1200,448.3590)
      and (2486.3100,446.6600) .. (2491.6800,442.0200) -- (2475.8200,420.5510) ..
      controls (2469.2400,426.6480) and (2462.1500,430.3090) .. (2456.0500,430.3090)
      .. controls (2441.4300,430.3090) and (2434.8200,417.6210) ..
      (2434.8200,388.3400) .. controls (2434.8200,371.9880) and (2437.0300,362.2380)
      .. (2441.1700,356.8710) .. controls (2444.5900,352.4800) and
      (2450.2000,349.7890) .. (2455.8200,349.7890) .. controls (2463.3800,349.7890)
      and (2470.2100,352.9610) .. (2478.2600,360.0390) -- (2480.2100,361.7500) --
      (2495.1000,341.9800) .. controls (2490.2100,337.1020) and (2488.0300,335.3980)
      .. (2483.6300,332.7110);
    \path[fill=ca69788,nonzero rule] (2586.3100,327.8320) -- (2586.3100,411.7620) ..
      controls (2586.3100,424.2110) and (2582.9100,428.8400) .. (2573.6300,428.8400)
      .. controls (2565.5900,428.8400) and (2555.0800,423.9610) ..
      (2547.7700,417.3790) -- (2547.7700,327.8320) -- (2514.5900,327.8320) --
      (2514.5900,472.2700) .. controls (2514.5900,483.9800) and (2513.6100,495.6990)
      .. (2512.1500,500.8200) -- (2545.5700,508.6290) .. controls
      (2547.0300,501.8010) and (2547.7700,490.0820) .. (2547.7700,478.1290) --
      (2547.7700,453.2380) .. controls (2547.7700,449.3400) and (2547.2900,444.2110)
      .. (2547.2900,442.7500) .. controls (2555.8200,450.8010) and
      (2569.9600,456.1720) .. (2582.6600,456.1720) .. controls (2598.5200,456.1720)
      and (2611.2100,449.3400) .. (2615.1000,438.3590) .. controls
      (2617.5400,431.2810) and (2618.2800,427.1290) .. (2618.2800,415.1800) --
      (2618.2800,327.8320) -- (2586.3100,327.8320);
    \path[fill=ca69788,nonzero rule] (2722.4600,405.4220) .. controls
      (2722.4600,414.6910) and (2721.4800,419.5700) .. (2718.5500,424.2110) ..
      controls (2715.3900,429.0900) and (2710.7400,431.5310) .. (2704.1600,431.5310)
      .. controls (2691.7200,431.5310) and (2684.6500,421.7700) ..
      (2684.6500,404.4410) -- (2684.6500,403.9610) -- (2722.4600,403.9610) --
      (2722.4600,405.4220) -- cycle(2684.1600,380.0390) -- (2684.1600,379.0700) ..
      controls (2684.1600,359.8010) and (2693.6700,348.8090) .. (2710.5100,348.8090)
      .. controls (2721.7400,348.8090) and (2732.2300,352.9610) ..
      (2742.2300,361.2620) -- (2754.9200,341.7380) .. controls (2740.5100,330.0310)
      and (2725.3900,324.4100) .. (2707.8300,324.4100) .. controls
      (2671.9500,324.4100) and (2648.7700,349.7890) .. (2648.7700,389.0700) ..
      controls (2648.7700,411.5200) and (2653.4200,426.3980) .. (2664.3900,438.6020)
      .. controls (2674.6500,450.0700) and (2687.0900,455.4410) ..
      (2703.6700,455.4410) .. controls (2718.0700,455.4410) and (2731.7400,450.5590)
      .. (2739.7900,442.2620) .. controls (2751.2500,430.5510) and
      (2756.3700,413.7190) .. (2756.3700,387.6090) -- (2756.3700,380.0390) --
      (2684.1600,380.0390);
    \path[fill=ca69788,nonzero rule] (2924.5100,327.8320) -- (2924.5100,403.4690) --
      (2875.7000,403.4690) -- (2875.7000,327.8320) -- (2841.8000,327.8320) --
      (2841.8000,497.3980) -- (2875.7000,497.3980) -- (2875.7000,431.5310) --
      (2924.5100,431.5310) -- (2924.5100,497.3980) -- (2959.1400,497.3980) --
      (2959.1400,327.8320) -- (2924.5100,327.8320);
    \path[fill=ca69788,nonzero rule] (3059.2000,424.4490) .. controls
      (3056.0200,428.6020) and (3050.9000,431.0390) .. (3045.0400,431.0390) ..
      controls (3037.2300,431.0390) and (3030.9000,426.1600) .. (3028.2000,418.3520)
      .. controls (3026.0200,411.7620) and (3024.7900,402.9800) ..
      (3024.7900,390.5390) .. controls (3024.7900,376.1410) and (3026.2500,365.4100)
      .. (3028.9500,359.0590) .. controls (3031.8800,352.2300) and
      (3039.1800,348.8090) .. (3045.5300,348.8090) .. controls (3059.6900,348.8090)
      and (3065.7800,361.5000) .. (3065.7800,391.0310) .. controls
      (3065.7800,407.8590) and (3063.5900,418.8400) .. (3059.2000,424.4490) --
      cycle(3085.7800,342.4690) .. controls (3076.2700,331.7300) and
      (3063.8300,325.1480) .. (3044.5500,325.1480) .. controls (3010.6400,325.1480)
      and (2988.4400,350.5200) .. (2988.4400,389.8010) .. controls
      (2988.4400,429.0900) and (3010.8800,455.1990) .. (3044.5500,455.1990) ..
      controls (3062.3600,455.1990) and (3076.2700,449.0900) .. (3087.0100,436.4100)
      .. controls (3097.0100,424.6990) and (3101.4100,411.0310) ..
      (3101.4100,390.7810) .. controls (3101.4100,369.3090) and (3096.5200,354.6720)
      .. (3085.7800,342.4690);
    \path[fill=ca69788,nonzero rule] (3208.2600,332.7110) .. controls
      (3199.4700,327.5900) and (3189.7300,324.8980) .. (3177.7700,324.8980) ..
      controls (3143.3600,324.8980) and (3123.3600,348.8090) .. (3123.3600,389.3200)
      .. controls (3123.3600,418.1090) and (3134.3400,437.1410) ..
      (3148.9800,447.1410) .. controls (3157.2700,452.7500) and (3169.4700,456.4220)
      .. (3179.9600,456.4220) .. controls (3188.2600,456.4220) and
      (3197.2900,454.4610) .. (3204.1200,450.8010) .. controls (3208.7500,448.3590)
      and (3210.9600,446.6600) .. (3216.3100,442.0200) -- (3200.4500,420.5510) ..
      controls (3193.8700,426.6480) and (3186.8000,430.3090) .. (3180.7000,430.3090)
      .. controls (3166.0500,430.3090) and (3159.4700,417.6210) ..
      (3159.4700,388.3400) .. controls (3159.4700,371.9880) and (3161.6600,362.2380)
      .. (3165.8200,356.8710) .. controls (3169.2200,352.4800) and
      (3174.8400,349.7890) .. (3180.4500,349.7890) .. controls (3188.0100,349.7890)
      and (3194.8400,352.9610) .. (3202.8900,360.0390) -- (3204.8400,361.7500) --
      (3219.7300,341.9800) .. controls (3214.8600,337.1020) and (3212.6600,335.3980)
      .. (3208.2600,332.7110);
    \path[fill=ca69788,nonzero rule] (3310.9600,327.8320) -- (3310.9600,411.7620) ..
      controls (3310.9600,424.2110) and (3307.5400,428.8400) .. (3298.2600,428.8400)
      .. controls (3290.2100,428.8400) and (3279.7300,423.9610) ..
      (3272.4000,417.3790) -- (3272.4000,327.8320) -- (3239.2200,327.8320) --
      (3239.2200,472.2700) .. controls (3239.2200,483.9800) and (3238.2400,495.6990)
      .. (3236.7800,500.8200) -- (3270.2100,508.6290) .. controls
      (3271.6800,501.8010) and (3272.4000,490.0820) .. (3272.4000,478.1290) --
      (3272.4000,453.2380) .. controls (3272.4000,449.3400) and (3271.9100,444.2110)
      .. (3271.9100,442.7500) .. controls (3280.4500,450.8010) and
      (3294.6100,456.1720) .. (3307.3000,456.1720) .. controls (3323.1600,456.1720)
      and (3335.8400,449.3400) .. (3339.7500,438.3590) .. controls
      (3342.1900,431.2810) and (3342.9100,427.1290) .. (3342.9100,415.1800) --
      (3342.9100,327.8320) -- (3310.9600,327.8320);
    \path[fill=ca69788,nonzero rule] (3420.2700,323.9300) .. controls
      (3404.9000,323.9300) and (3388.0700,328.8090) .. (3369.7700,338.3320) --
      (3381.4800,362.2380) .. controls (3391.4800,356.1290) and (3408.8100,348.0820)
      .. (3422.9500,348.0820) .. controls (3432.2300,348.0820) and
      (3439.5500,354.1800) .. (3439.5500,362.2380) .. controls (3439.5500,370.7810)
      and (3433.4600,375.1720) .. (3420.2700,377.6090) -- (3405.6300,380.2890) ..
      controls (3397.3400,381.7500) and (3387.0900,387.6090) .. (3382.7000,392.9800)
      .. controls (3378.3000,398.3400) and (3375.6300,407.3710) ..
      (3375.6300,415.4220) .. controls (3375.6300,439.8200) and (3394.9000,456.1720)
      .. (3423.9300,456.1720) .. controls (3443.9500,456.1720) and
      (3457.1100,450.0700) .. (3468.5700,444.4610) -- (3457.8500,422.5000) ..
      controls (3445.4100,428.8400) and (3436.3700,431.5310) .. (3427.1100,431.5310)
      .. controls (3417.6000,431.5310) and (3411.2500,426.6480) ..
      (3411.2500,419.3320) .. controls (3411.2500,412.9800) and (3415.3900,409.5700)
      .. (3426.8600,406.6410) -- (3441.9900,402.7300) .. controls
      (3457.3600,398.8320) and (3462.4800,394.1910) .. (3466.8800,388.5820) ..
      controls (3471.5000,382.7300) and (3473.7100,375.6480) .. (3473.7100,367.3590)
      .. controls (3473.7100,341.4880) and (3452.2300,323.9300) ..
      (3420.2700,323.9300);
    \path[fill=ca69788,nonzero rule] (3578.1100,332.7110) .. controls
      (3569.3200,327.5900) and (3559.5700,324.8980) .. (3547.6200,324.8980) ..
      controls (3513.2000,324.8980) and (3493.2000,348.8090) .. (3493.2000,389.3200)
      .. controls (3493.2000,418.1090) and (3504.1800,437.1410) ..
      (3518.8300,447.1410) .. controls (3527.1100,452.7500) and (3539.3200,456.4220)
      .. (3549.8000,456.4220) .. controls (3558.1100,456.4220) and
      (3567.1300,454.4610) .. (3573.9600,450.8010) .. controls (3578.5900,448.3590)
      and (3580.8000,446.6600) .. (3586.1500,442.0200) -- (3570.2900,420.5510) ..
      controls (3563.7100,426.6480) and (3556.6400,430.3090) .. (3550.5500,430.3090)
      .. controls (3535.9000,430.3090) and (3529.3200,417.6210) ..
      (3529.3200,388.3400) .. controls (3529.3200,371.9880) and (3531.5000,362.2380)
      .. (3535.6600,356.8710) .. controls (3539.0600,352.4800) and
      (3544.6900,349.7890) .. (3550.2900,349.7890) .. controls (3557.8500,349.7890)
      and (3564.6900,352.9610) .. (3572.7300,360.0390) -- (3574.6900,361.7500) --
      (3589.5700,341.9800) .. controls (3584.7100,337.1020) and (3582.5000,335.3980)
      .. (3578.1100,332.7110);
    \path[fill=ca69788,nonzero rule] (3680.7800,327.8320) -- (3680.7800,411.7620) ..
      controls (3680.7800,424.2110) and (3677.3800,428.8400) .. (3668.1100,428.8400)
      .. controls (3660.0600,428.8400) and (3649.5500,423.9610) ..
      (3642.2500,417.3790) -- (3642.2500,327.8320) -- (3609.0600,327.8320) --
      (3609.0600,472.2700) .. controls (3609.0600,483.9800) and (3608.0900,495.6990)
      .. (3606.6200,500.8200) -- (3640.0400,508.6290) .. controls
      (3641.5000,501.8010) and (3642.2500,490.0820) .. (3642.2500,478.1290) --
      (3642.2500,453.2380) .. controls (3642.2500,449.3400) and (3641.7600,444.2110)
      .. (3641.7600,442.7500) .. controls (3650.2900,450.8010) and
      (3664.4300,456.1720) .. (3677.1300,456.1720) .. controls (3692.9900,456.1720)
      and (3705.6800,449.3400) .. (3709.5700,438.3590) .. controls
      (3712.0100,431.2810) and (3712.7500,427.1290) .. (3712.7500,415.1800) --
      (3712.7500,327.8320) -- (3680.7800,327.8320);
    \path[fill=ca69788,nonzero rule] (3831.5800,324.4100) .. controls
      (3827.4200,327.3400) and (3824.0000,331.4880) .. (3821.8200,336.6210) ..
      controls (3813.7700,328.8090) and (3802.0500,324.6600) .. (3788.8700,324.6600)
      .. controls (3771.3100,324.6600) and (3756.1900,332.9490) ..
      (3752.0300,344.9100) .. controls (3750.0800,350.5200) and (3749.3600,357.1090)
      .. (3749.3600,369.8010) -- (3749.3600,449.8200) -- (3781.5600,455.9300) --
      (3781.5600,375.6480) .. controls (3781.5600,364.4300) and (3782.5400,358.5700)
      .. (3784.2400,355.1600) .. controls (3785.9600,351.7380) and
      (3790.8200,349.3010) .. (3795.7000,349.3010) .. controls (3803.7500,349.3010)
      and (3813.5200,355.1600) .. (3815.9600,361.2620) -- (3815.9600,449.0900) --
      (3847.1900,455.6800) -- (3847.1900,360.2810) .. controls (3847.1900,351.9880)
      and (3849.8800,343.4490) .. (3854.7500,337.5900) -- (3831.5800,324.4100);
    \path[fill=ca69788,nonzero rule] (3915.7200,324.8980) .. controls
      (3901.8200,324.8980) and (3890.5900,331.4880) .. (3886.7000,341.9800) ..
      controls (3884.2600,348.3320) and (3883.7700,352.2300) .. (3883.7700,370.0390)
      -- (3883.7700,463.2500) .. controls (3883.7700,479.5900) and
      (3883.2800,489.5900) .. (3882.0500,500.8200) -- (3915.4900,508.3790) ..
      controls (3916.7000,501.5510) and (3917.1900,493.5000) .. (3917.1900,475.9300)
      -- (3917.1900,378.5820) .. controls (3917.1900,357.1090) and
      (3917.4400,354.1800) .. (3919.3900,350.7700) .. controls (3920.6100,348.5700)
      and (3923.2800,347.3520) .. (3925.9800,347.3520) .. controls
      (3927.1900,347.3520) and (3927.9300,347.3520) .. (3929.6300,347.8400) --
      (3935.2500,328.3200) .. controls (3929.6300,326.1210) and (3922.7900,324.8980)
      .. (3915.7200,324.8980);
    \path[fill=ca69788,nonzero rule] (4021.1100,405.4220) .. controls
      (4021.1100,414.6910) and (4020.1400,419.5700) .. (4017.2100,424.2110) ..
      controls (4014.0200,429.0900) and (4009.3900,431.5310) .. (4002.8100,431.5310)
      .. controls (3990.3700,431.5310) and (3983.2800,421.7700) ..
      (3983.2800,404.4410) -- (3983.2800,403.9610) -- (4021.1100,403.9610) --
      (4021.1100,405.4220) -- cycle(3982.7900,380.0390) -- (3982.7900,379.0700) ..
      controls (3982.7900,359.8010) and (3992.3000,348.8090) .. (4009.1400,348.8090)
      .. controls (4020.3700,348.8090) and (4030.8600,352.9610) ..
      (4040.8600,361.2620) -- (4053.5500,341.7380) .. controls (4039.1600,330.0310)
      and (4024.0200,324.4100) .. (4006.4600,324.4100) .. controls
      (3970.6100,324.4100) and (3947.4200,349.7890) .. (3947.4200,389.0700) ..
      controls (3947.4200,411.5200) and (3952.0500,426.3980) .. (3963.0300,438.6020)
      .. controls (3973.2800,450.0700) and (3985.7200,455.4410) ..
      (4002.3200,455.4410) .. controls (4016.7200,455.4410) and (4030.3700,450.5590)
      .. (4038.4400,442.2620) .. controls (4049.9000,430.5510) and
      (4055.0200,413.7190) .. (4055.0200,387.6090) -- (4055.0200,380.0390) --
      (3982.7900,380.0390);
    \path[fill=cf68712,nonzero rule] (1253.2400,150.5390) .. controls
      (1251.5300,158.3520) and (1246.4100,180.5510) .. (1246.4100,180.5510) ..
      controls (1246.4100,180.5510) and (1241.5300,160.5510) .. (1238.3600,147.8710)
      .. controls (1235.1900,135.6600) and (1232.9900,127.6090) ..
      (1229.3300,116.3910) -- (1262.5100,116.3910) .. controls (1262.5100,116.3910)
      and (1256.9000,134.1910) .. (1253.2400,150.5390) -- cycle(1282.7600,47.8320)
      -- (1270.8100,88.0820) -- (1221.0400,88.0820) -- (1209.0800,47.8320) --
      (1173.4600,47.8320) -- (1229.0900,217.8910) -- (1265.9300,217.8910) --
      (1319.3700,47.8320) -- (1282.7600,47.8320);
    \path[fill=cf68712,nonzero rule] (1465.7600,47.8320) -- (1465.7600,130.3010) ..
      controls (1465.7600,145.1800) and (1464.0600,148.1090) .. (1455.5200,148.1090)
      .. controls (1449.4200,148.1090) and (1440.8800,143.9610) ..
      (1433.8000,137.6090) -- (1433.8000,47.8320) -- (1402.8200,47.8320) --
      (1402.8200,129.0820) .. controls (1402.8200,144.6910) and (1400.6200,148.3520)
      .. (1391.5900,148.3520) .. controls (1385.4900,148.3520) and
      (1377.2000,145.1800) .. (1370.1200,138.8320) -- (1370.1200,47.8320) --
      (1338.1600,47.8320) -- (1338.1600,134.9300) .. controls (1338.1600,152.9800)
      and (1336.9400,160.8010) .. (1333.5200,166.8910) -- (1363.0500,174.9490) ..
      controls (1365.2400,171.5310) and (1366.2200,168.6020) .. (1367.4400,162.2620)
      .. controls (1375.9800,170.5510) and (1386.4700,174.9490) ..
      (1397.9300,174.9490) .. controls (1408.1800,174.9490) and (1416.7200,171.5310)
      .. (1423.3100,164.6990) .. controls (1425.0200,162.9880) and
      (1426.7200,160.8010) .. (1428.1900,158.6020) .. controls (1439.6600,170.3090)
      and (1449.9100,174.9490) .. (1463.5700,174.9490) .. controls
      (1473.3300,174.9490) and (1482.6000,172.0200) .. (1488.2100,167.1290) ..
      controls (1495.2900,161.0390) and (1497.4800,153.7190) .. (1497.4800,136.6410)
      -- (1497.4800,47.8320) -- (1465.7600,47.8320);
    \path[fill=cf68712,nonzero rule] (1604.5900,132.9800) .. controls
      (1602.1600,143.7190) and (1595.3300,148.8400) .. (1586.7900,148.8400) ..
      controls (1578.2500,148.8400) and (1572.4000,143.4690) .. (1568.2500,138.8320)
      -- (1568.2500,81.2500) .. controls (1572.6400,77.3516) and (1577.7600,73.1992)
      .. (1586.5500,73.1992) .. controls (1594.5900,73.1992) and (1599.7200,76.3711)
      .. (1602.4100,82.9609) .. controls (1605.3300,89.7891) and (1606.0600,96.3789)
      .. (1606.0600,111.7500) .. controls (1606.0600,122.9690) and
      (1605.8200,127.6090) .. (1604.5900,132.9800) -- cycle(1619.9700,55.1484) ..
      controls (1611.4300,48.3203) and (1602.9000,45.3906) .. (1592.6400,45.3906) ..
      controls (1580.4400,45.3906) and (1571.9100,49.0508) .. (1564.8300,56.8516) ..
      controls (1563.8500,52.4688) and (1563.6100,51.2500) .. (1561.4100,47.8320) --
      (1532.1300,47.8320) .. controls (1534.5800,53.4414) and (1535.3100,57.0898) ..
      (1535.3100,74.4219) -- (1535.3100,190.0700) .. controls (1535.3100,205.4410)
      and (1534.8200,213.7420) .. (1533.1100,220.8200) -- (1566.5400,228.6210) ..
      controls (1567.5100,220.0820) and (1567.7600,214.9610) .. (1567.7600,206.4220)
      -- (1567.7600,176.8980) .. controls (1567.7600,173.2380) and
      (1567.5100,167.8710) .. (1567.0300,165.9100) .. controls (1573.3600,172.7500)
      and (1582.1500,175.9220) .. (1593.6200,175.9220) .. controls
      (1624.1200,175.9220) and (1643.1500,151.5200) .. (1643.1500,112.4920) ..
      controls (1643.1500,87.1016) and (1635.3400,67.3516) .. (1619.9700,55.1484);
    \path[fill=cf68712,nonzero rule] (1740.0100,125.4220) .. controls
      (1740.0100,134.6800) and (1739.0400,139.5700) .. (1736.1100,144.1990) ..
      controls (1732.9300,149.0900) and (1728.3000,151.5200) .. (1721.7100,151.5200)
      .. controls (1709.2700,151.5200) and (1702.1900,141.7620) ..
      (1702.1900,124.4410) -- (1702.1900,123.9490) -- (1740.0100,123.9490) --
      (1740.0100,125.4220) -- cycle(1701.7000,100.0390) -- (1701.7000,99.0703) ..
      controls (1701.7000,79.7891) and (1711.2200,68.8086) .. (1728.0500,68.8086) ..
      controls (1739.2800,68.8086) and (1749.7700,72.9492) .. (1759.7700,81.2500) --
      (1772.4700,61.7305) .. controls (1758.0700,50.0195) and (1742.9300,44.4102) ..
      (1725.3800,44.4102) .. controls (1689.5100,44.4102) and (1666.3200,69.7891) ..
      (1666.3200,109.0700) .. controls (1666.3200,131.5200) and (1670.9600,146.3980)
      .. (1681.9400,158.6020) .. controls (1692.1900,170.0590) and
      (1704.6300,175.4300) .. (1721.2300,175.4300) .. controls (1735.6200,175.4300)
      and (1749.2800,170.5510) .. (1757.3400,162.2620) .. controls
      (1768.8000,150.5390) and (1773.9200,133.7190) .. (1773.9200,107.6020) --
      (1773.9200,100.0390) -- (1701.7000,100.0390);
    \path[fill=cf68712,nonzero rule] (1862.9900,144.1990) .. controls
      (1859.8200,145.9100) and (1857.1300,146.6480) .. (1853.4800,146.6480) ..
      controls (1846.1600,146.6480) and (1839.5700,143.2300) .. (1833.4700,136.1480)
      -- (1833.4700,47.8320) -- (1800.7800,47.8320) -- (1800.7800,131.2700) ..
      controls (1800.7800,148.1090) and (1798.8200,160.8010) .. (1796.1400,167.8710)
      -- (1825.4200,175.6800) .. controls (1828.3400,170.5510) and
      (1830.0500,164.9490) .. (1830.5400,157.8710) .. controls (1837.6200,167.3790)
      and (1847.6200,175.6800) .. (1859.8200,175.6800) .. controls
      (1864.7000,175.6800) and (1866.9000,175.1910) .. (1872.0200,172.9880) --
      (1862.9900,144.1990);
    \path[fill=cf68712,nonzero rule] (1931.5300,151.0310) .. controls
      (1919.0900,151.0310) and (1912.0100,144.1990) .. (1912.0100,132.5000) ..
      controls (1912.0100,119.8010) and (1919.8200,114.6800) .. (1931.2800,114.6800)
      .. controls (1944.2100,114.6800) and (1951.2900,121.0200) ..
      (1951.2900,132.5000) .. controls (1951.2900,144.1990) and (1943.9700,151.0310)
      .. (1931.5300,151.0310) -- cycle(1986.1800,149.8090) .. controls
      (1983.0100,149.8090) and (1979.5900,150.3010) .. (1977.6400,150.5390) ..
      controls (1982.5200,144.6910) and (1984.9700,138.3400) .. (1984.9700,130.0510)
      .. controls (1984.9700,108.5820) and (1965.4400,92.9609) ..
      (1938.8400,92.9609) .. controls (1937.3900,92.9609) and (1936.4100,92.9609) ..
      (1933.9700,93.2109) .. controls (1925.6700,89.3008) and (1921.0400,86.3789) ..
      (1921.0400,82.9609) .. controls (1921.0400,81.2500) and (1922.9900,80.0313) ..
      (1926.4000,80.0313) -- (1943.9700,79.7891) .. controls (1963.0000,79.5508) and
      (1973.0000,76.6211) .. (1981.7900,68.5703) .. controls (1989.1100,61.7305) and
      (1992.5200,53.1992) .. (1992.5200,41.9688) .. controls (1992.5200,31.4805) and
      (1989.3500,23.4297) .. (1982.5200,16.1094) .. controls (1971.5400,4.3984) and
      (1952.7600,0.0000) .. (1933.2400,0.0000) .. controls (1915.4200,0.0000) and
      (1897.1200,2.9297) .. (1885.9000,13.4219) .. controls (1879.0700,19.7695) and
      (1875.6600,27.0898) .. (1875.6600,35.6289) .. controls (1875.6600,42.4609) and
      (1877.3600,45.8711) .. (1878.5800,48.3203) -- (1908.8300,48.3203) .. controls
      (1907.6100,45.3906) and (1907.3700,43.6797) .. (1907.3700,40.2617) .. controls
      (1907.3700,30.2617) and (1915.6700,24.8906) .. (1930.8000,24.8906) .. controls
      (1939.0900,24.8906) and (1946.1700,25.8594) .. (1951.2900,29.0391) .. controls
      (1956.1700,31.9688) and (1959.3400,36.6094) .. (1959.3400,41.7188) .. controls
      (1959.3400,52.9492) and (1949.3400,56.3594) .. (1936.4100,56.6094) --
      (1922.5000,56.8516) .. controls (1907.8600,57.0898) and (1898.3400,58.3203) ..
      (1892.4900,60.7617) .. controls (1886.6400,62.9492) and (1882.9700,68.5703) ..
      (1882.9700,77.1016) .. controls (1882.9700,85.1602) and (1885.4200,92.7188) ..
      (1905.9100,98.0898) .. controls (1887.8600,102.7190) and (1879.0700,114.1910)
      .. (1879.0700,132.7380) .. controls (1879.0700,158.3520) and
      (1899.8100,174.9490) .. (1931.7700,174.9490) .. controls (1938.8400,174.9490)
      and (1945.1900,173.9690) .. (1954.2200,171.7700) .. controls
      (1961.0500,170.0590) and (1965.4400,169.0900) .. (1969.5900,169.0900) ..
      controls (1978.6200,169.0900) and (1987.8900,172.9880) .. (1995.4500,179.8320)
      -- (2009.1200,159.0900) .. controls (2002.0400,152.5000) and
      (1995.2100,149.8090) .. (1986.1800,149.8090);
    \path[fill=cf68712,nonzero rule] (2015.4300,100.5310) -- (2074.0003,100.5310) --
      (2074.0003,128.8318) -- (2015.4300,128.8318) -- (2015.4300,100.5310) -- cycle;
    \path[fill=cf68712,nonzero rule] (2245.7200,46.3594) -- (2208.3800,46.3594) --
      (2193.2600,115.4220) .. controls (2188.3800,136.6410) and (2185.6800,157.6210)
      .. (2185.2000,163.7300) .. controls (2184.7300,159.8200) and
      (2181.8000,138.3400) .. (2177.1500,115.8980) -- (2163.0100,46.3594) --
      (2124.2000,46.3594) -- (2083.7100,217.3980) -- (2119.8200,217.3980) --
      (2134.4600,150.5390) .. controls (2141.2800,118.8200) and (2143.4800,94.6719)
      .. (2143.4800,94.6719) .. controls (2144.2100,101.2620) and
      (2147.1500,125.4220) .. (2152.2700,148.8400) -- (2167.4000,217.3980) --
      (2204.7300,217.3980) -- (2220.3300,141.5200) .. controls (2223.5200,125.6600)
      and (2227.6600,96.6289) .. (2227.6600,96.6289) .. controls
      (2228.1400,102.0000) and (2233.2600,136.3910) .. (2236.9300,152.9800) --
      (2250.8400,217.3980) -- (2286.7000,217.3980) -- (2245.7200,46.3594);
    \path[fill=cf68712,nonzero rule] (2366.5000,125.4220) .. controls
      (2366.5000,134.6800) and (2365.5300,139.5700) .. (2362.6000,144.1990) ..
      controls (2359.4100,149.0900) and (2354.7900,151.5200) .. (2348.2000,151.5200)
      .. controls (2335.7600,151.5200) and (2328.6700,141.7620) ..
      (2328.6700,124.4410) -- (2328.6700,123.9490) -- (2366.5000,123.9490) --
      (2366.5000,125.4220) -- cycle(2328.1800,100.0390) -- (2328.1800,99.0703) ..
      controls (2328.1800,79.7891) and (2337.7000,68.8086) .. (2354.5300,68.8086) ..
      controls (2365.7600,68.8086) and (2376.2500,72.9492) .. (2386.2500,81.2500) --
      (2398.9500,61.7305) .. controls (2384.5500,50.0195) and (2369.4100,44.4102) ..
      (2351.8600,44.4102) .. controls (2316.0000,44.4102) and (2292.8100,69.7891) ..
      (2292.8100,109.0700) .. controls (2292.8100,131.5200) and (2297.4400,146.3980)
      .. (2308.4200,158.6020) .. controls (2318.6700,170.0590) and
      (2331.1100,175.4300) .. (2347.7100,175.4300) .. controls (2362.1100,175.4300)
      and (2375.7600,170.5510) .. (2383.8300,162.2620) .. controls
      (2395.2900,150.5390) and (2400.4100,133.7190) .. (2400.4100,107.6020) --
      (2400.4100,100.0390) -- (2328.1800,100.0390);
    \path[fill=cf68712,nonzero rule] (2427.2700,47.8320) -- (2427.2700,170.5510) --
      (2459.9600,175.6800) -- (2459.9600,47.8320) -- (2427.2700,47.8320) --
      cycle(2443.6100,186.8980) .. controls (2432.6400,186.8980) and
      (2423.5900,195.9300) .. (2423.5900,207.1480) .. controls (2423.5900,218.3710)
      and (2432.8700,227.4100) .. (2444.1000,227.4100) .. controls
      (2455.0800,227.4100) and (2463.8700,218.3710) .. (2463.8700,207.1480) ..
      controls (2463.8700,195.9300) and (2454.8200,186.8980) ..
      (2443.6100,186.8980);
    \path[fill=cf68712,nonzero rule] (2565.6100,139.8090) .. controls
      (2559.5100,145.1800) and (2553.6500,147.8710) .. (2547.7900,147.8710) ..
      controls (2533.1600,147.8710) and (2526.8200,135.6600) .. (2526.8200,107.8520)
      .. controls (2526.8200,81.0117) and (2532.4200,72.2188) .. (2549.5100,72.2188)
      .. controls (2555.6100,72.2188) and (2562.4400,76.3711) .. (2565.6100,80.2813)
      -- (2565.6100,139.8090) -- cycle(2572.9300,47.8320) .. controls
      (2571.9500,49.7813) and (2571.4600,51.7305) .. (2570.9800,55.1484) .. controls
      (2562.9300,48.0703) and (2553.4000,44.6484) .. (2542.1900,44.6484) .. controls
      (2510.4700,44.6484) and (2490.4500,69.3008) .. (2490.4500,108.0900) ..
      controls (2490.4500,147.1290) and (2512.1700,173.9690) .. (2543.8900,173.9690)
      .. controls (2552.9100,173.9690) and (2560.0000,171.7700) ..
      (2566.0900,166.8910) .. controls (2565.6100,169.5820) and (2565.1200,178.1210)
      .. (2565.1200,185.4300) -- (2565.1200,228.3710) -- (2597.5800,223.2620) --
      (2597.5800,93.4492) .. controls (2597.5800,62.7109) and (2600.0000,52.7109) ..
      (2602.2100,47.8320) -- (2572.9300,47.8320);
    \path[fill=cf68712,nonzero rule] (2701.7600,125.4220) .. controls
      (2701.7600,134.6800) and (2700.7800,139.5700) .. (2697.8500,144.1990) ..
      controls (2694.6900,149.0900) and (2690.0400,151.5200) .. (2683.4600,151.5200)
      .. controls (2671.0200,151.5200) and (2663.9500,141.7620) ..
      (2663.9500,124.4410) -- (2663.9500,123.9490) -- (2701.7600,123.9490) --
      (2701.7600,125.4220) -- cycle(2663.4600,100.0390) -- (2663.4600,99.0703) ..
      controls (2663.4600,79.7891) and (2672.9700,68.8086) .. (2689.8000,68.8086) ..
      controls (2701.0400,68.8086) and (2711.5200,72.9492) .. (2721.5200,81.2500) --
      (2734.2200,61.7305) .. controls (2719.8000,50.0195) and (2704.6900,44.4102) ..
      (2687.1300,44.4102) .. controls (2651.2500,44.4102) and (2628.0700,69.7891) ..
      (2628.0700,109.0700) .. controls (2628.0700,131.5200) and (2632.7100,146.3980)
      .. (2643.6900,158.6020) .. controls (2653.9500,170.0590) and
      (2666.3900,175.4300) .. (2682.9700,175.4300) .. controls (2697.3600,175.4300)
      and (2711.0400,170.5510) .. (2719.0800,162.2620) .. controls
      (2730.5500,150.5390) and (2735.6600,133.7190) .. (2735.6600,107.6020) --
      (2735.6600,100.0390) -- (2663.4600,100.0390);
    \path[fill=cf68712,nonzero rule] (2828.8900,47.8320) -- (2828.8900,129.0820) ..
      controls (2828.8900,143.2300) and (2826.4500,147.3790) .. (2817.8900,147.3790)
      .. controls (2811.3100,147.3790) and (2802.7700,142.9800) ..
      (2795.2100,136.1480) -- (2795.2100,47.8320) -- (2762.5200,47.8320) --
      (2762.5200,138.3400) .. controls (2762.5200,149.0900) and (2761.0500,159.3320)
      .. (2758.1300,167.6210) -- (2787.1500,175.9220) .. controls
      (2790.0800,170.8010) and (2791.8000,165.4220) .. (2791.8000,160.3090) ..
      controls (2796.6800,163.7300) and (2800.8200,166.6480) .. (2806.1900,169.5820)
      .. controls (2812.7700,172.9880) and (2821.3100,174.9490) ..
      (2828.6300,174.9490) .. controls (2842.5400,174.9490) and (2854.7500,167.6210)
      .. (2858.6500,156.8910) .. controls (2860.3500,152.2500) and
      (2861.0900,146.8910) .. (2861.0900,139.0820) -- (2861.0900,47.8320) --
      (2828.8900,47.8320);
    \path[fill=cf68712,nonzero rule] (3877.2100,2075.6300) .. controls
      (3820.4500,2075.6300) and (3764.2700,2050.5500) .. (3726.4300,2002.5000) --
      (3139.9800,1257.8000) -- (2553.5400,2002.5000) .. controls
      (2517.1900,2048.6700) and (2461.6600,2075.6200) .. (2402.8900,2075.6200) --
      (2402.8900,2075.6200) .. controls (2344.1300,2075.6200) and
      (2288.6100,2048.6700) .. (2252.2500,2002.5000) -- (1665.8400,1257.8200) --
      (1079.4500,2002.5000) .. controls (1043.1100,2048.6700) and
      (987.5780,2075.6200) .. (928.8090,2075.6200) .. controls (870.0390,2075.6200)
      and (814.5120,2048.6700) .. (778.1640,2002.5000) -- (41.1133,1066.4900) ..
      controls (-24.4063,983.3010) and (-10.0703,862.7380) .. (73.1367,797.2190) ..
      controls (156.3280,731.6990) and (276.8950,746.0390) .. (342.4100,829.2380) --
      (928.8090,1573.9300) -- (1515.1900,829.2380) .. controls (1551.5400,783.0700)
      and (1607.0700,756.1210) .. (1665.8400,756.1210) .. controls
      (1724.6100,756.1210) and (1780.1300,783.0700) .. (1816.4800,829.2380) --
      (2402.9100,1573.9300) -- (2989.3400,829.2380) .. controls (3025.7000,783.0700)
      and (3081.2200,756.1210) .. (3139.9800,756.1210) .. controls
      (3198.7500,756.1210) and (3254.2800,783.0700) .. (3290.6300,829.2380) --
      (4027.7200,1765.2400) .. controls (4093.2400,1848.4400) and
      (4078.9100,1969.0000) .. (3995.7000,2034.5200) .. controls
      (3960.5700,2062.1900) and (3918.7300,2075.6300) .. (3877.2100,2075.6300);
    \path[fill=ca69788,nonzero rule] (700.9180,983.6210) .. controls
      (700.9180,1109.2700) and (802.7730,1211.1200) .. (928.4180,1211.1200) ..
      controls (1054.0600,1211.1200) and (1155.9200,1109.2700) ..
      (1155.9200,983.6210) .. controls (1155.9200,857.9690) and (1054.0600,756.1210)
      .. (928.4180,756.1210) .. controls (802.7730,756.1210) and (700.9180,857.9690)
      .. (700.9180,983.6210);
  \end{scope}
\end{scope}

\end{tikzpicture}
\end{minipage}
Bestätigung gemäß \S\ 12 APO\\
\rule[1ex]{\textwidth}{0.5pt}
\vspace*{0.5cm}
\begin{tabbing}
  Name und Vorname\\
  der Studentin/des Studenten:\qquad\=\textbf{\IhrNachname, \IhrVorname}\\[1cm]
  Studiengang:                      \>\textbf{\IhrStudiengang}
\end{tabbing}
\rule[1ex]{\textwidth}{0.5pt}
\vspace*{0.5cm}
Ich bestätige, dass ich die \IhreArbeit\ mit dem Titel:
\begin{center}
  \textbf{\IhrTitelDE}
\end{center}
\vspace*{0.5cm}
selbständig verfasst, noch nicht anderweitig für Prüfungszwecke vorgelegt, keine anderen als die angegebenen Quellen oder Hilfsmittel benützt sowie wörtliche und sinngemäße Zitate als solche gekennzeichnet habe.\\[0.5cm]
\rule[1ex]{\textwidth}{0.5pt}
\begin{tabbing}
  Datum:\hspace{2cm}\=\today\\[1cm]
  Unterschrift:\> 
\end{tabbing}
\rule[1ex]{\textwidth}{0.5pt}
\clearpage		% 3. Seite: Formblatt Bestaetigung nach Paragraph 12 APO
  \include{formblatt_summary}			% 4. Seite: Formblatt Zusammenfassung
  \tableofcontents
  \pagenumbering{arabic}
  \chapter{Einleitung}
Die Krones AG bietet Anlagen für die Getränkeindustrie als auch 
Nahrungsmittelhersteller, von der Prozesstechnik bis hin zur IT-Lösung. 
Die Komplettlinie beeinhaltet auch das bereitstellen von Software auf den einzelnen Produktionsanlagen. 
Hierfür werden eine Vielzahl von Produktionslinienanwendungen auf den Anlagen installiert, gewartet
und verwaltet. Ein riesiger Aufwand der Fehleranfälligkeiten wie fehlende Frameworks, Bibliotheken
und anderer Abhängigkeiten mit sich bringt.
Eigene Server müssen für die Kommunikation der Anlagen verbaut und gewartet werden,
was zusätzlich Ressourcen beansprucht und automatisch die Kosten für die Inbetriebnahme einer solchen
Linien erhöhen. Die Weiterentwicklung der zukünftigen Bereitstellung von Produktionsanlagensoftware
erfolgt mithilfe eines Proof of Concept (PoC), welcher die Möglichkeiten einer wartungsfreien Infrastruktur
durch ein continuous delivery System evaluiert. Dies verläuft in Zusammenarbeit mit dem
Kooperationspartner und Softwarteunternehmen SUSE GmbH, welches das wartungsfreie Betriebssystem
SUSE Linux Enterprise Micro und die multi-cluster Orchestrierungsplattform Rancher anbietet.

Als Grundlage hierfür dient das Open-Source-System Kubernetes, welches zur Automatisierung, Skalierung
und Verwaltung von containerisierten Anwendungen bestimmt ist. Künftige Produktionsanlagen sollen mittels zusätzlichen Edge Devices
als Knotenpunkte in einem Kubernetes Cluster fungieren, Ressourcen teilen, untereinander kommunizieren und Softwarepakete unkompliziert bereitstellen.
Die Integration der kompakten Linux Rechner ermöglichen den Variablen Einsatz von Hardwareressourcen beim Kunden, der je nach Leistungsanspruch Knotenpunkte erweitern kann.
Dabei soll es für die einzelnen Anwendungen möglich sein, sowohl auf cloudbasierten als auch auf on-premise Hardware zur Verfügung gestellt zu werden.
Ein hybrides Kubernetes Cluster ermöglicht es somit lokale Rechenleistung oder öffentliche Cloudressourcen in der selben Softwareumgebung zu nutzen.
\section{Motivation}
Die Vorteile von Kubernetes und dem stetigen Paradigmenwechsel der Softwarelandschaft im Cloudbereich, welcher
den Wechsel von monolithischen Architektur zu einer mehr flexibleren microservice Architektur
bevorzugt, sind das Hauptmotiv der Auswertung neuer agiler Distributionsmöglichkeiten.
Die Containerisierung von Anwendungen ermöglichen erst die Aufteilung großer Projekte
in kleine unabängige Services die mittels Orchestrierungsplattformen sinnvoll gebündelt werden können.
Namenhafte Unternehmen wie Netflix, Amazon und Uber entwickeln und verwenden
bereits robuste und komplexe Microservices die containerisiert auf Plattformen
verwaltet werden \cite{microservice}. 
%Absatz nochmal durchgehen

Durch die Flexibiltät einer solchen infrastuktur ist es möglich Anwendungsfälle im Bereich
der künstlichen Intelligenz
für die industrie zu testen. Die Anlage Linatronic AI der Krones AG nutzt bereits Deep-Learning-Technologie,
um in der Linie mittels Vollinspektion Schäden, Dichtflächen oder Seitenwanddicken
zu erkennen und Prozesse zu optimieren \cite{linatronic}. Allgemein sind Anwendungen mit künstlicher Intelligenz durch ihre Komplexität
und vielzahl an Abhängigkeiten schwierig zu entwickeln und bereitzustellen. 
Eine passende Plattform für Anwendungsfälle mit Bezug zur künstlichen Intelligenz
muss eine Vielzahl an Services anbieten. Verwaltung von Ressourcen wie Speicher,
Rechenleistung und Verbindungsgeschwindigkeit für die Datenübertragung, 
bei der Ausführung einzelner Phasen von der Datenverarbeitung bis hin zur Evaluierung und Entwicklung
\cite{mlops}. 

\section{Zielsetzung}
Ziel dieser Arbeit ist die Entwicklung einer Microservice
Architektur in einem hybriden Kubernetes Cluster. Das Endresultat
soll eine Anwendung werden die mittels einer Weboberfläche, welche über eine Domain erreichbar ist,
ein Login-Verfahren mittels einem backend Service ermöglichen der ein Authentifierzungsverfahren
per Gesichtserkennung verwendet. 
Diese Daten sollen schließlich verarbeitet und persistent 
gespeichert werden, um bei erneuten Aufruf der Website bestehen zu bleiben.
Die Konzeption der Anwendung findet containerisiert auf mehreren Software und Hardware Layern
statt. Beispielhaft an der Auswahl von Ressourcen zur Auswertung von Gesichtsdaten, dabei
steht die Option zwischen einem \glqq Fast\grqq{} oder \glqq Performance\grqq{} Modus zur Verfügung,
welcher entweder einen Knotenpunkt mit Grafikprozessor auswählt oder einer herkömmlichen Prozessoreinheit.
Das ganze System wird auf einem Kubernetes Cluster bereitgestellt und verwaltet.
Ein Ingress Controller dient dabei als Loadbalancer und verteilt die Last beim 
Aufrufen der Website und der Kommunikation zwischen backend Service.


  \chapter{Grundlagen}
Dieses Kapitel erläutert die Grundlagen die zum 
Verständnis dieser Arbeit notwendig sind. 
Angefangen mit der Erklärung für Technologien
Docker und Kubernetes, hinzu Microservices.

%TEXT!!!!

\section{Docker} \label{Docker}

In diesem Abschnitt wird die Technologie \glqq Docker\grqq{} näher erläutert und
nicht das Unternehmen \glqq Docker, Inc.\grqq{}, dass für die maßgebliche Entwicklung dessen verantwortlich ist.
Angefangen mit der Terminologie, zum deutlicheren Verständnis der nächsten Abschnitte.
Fortgesetzt mit der aufsteigenden Erklärung der Architektur bis zum Aufbau eines Containers.

%\subsection{Terminologie}
%\begin{itemize}
%    \item \textbf{Container}: Isolierter Prozess mit einer laufenden Anwendung.
%    \item \textbf{Volumes}: Persistente Daten eines Containers.
%    \item \textbf{Image}: Einheit mit ausführbaren Code, Abhängigkeiten und Betriebssystem. Aufgeteilt in mehreren Schichten.
%    \item \textbf{Dockerfile}: Textdatei zum erstellen eines Docker-Images.    
%\end{itemize}

\subsection{Architektur}
Die Docker Technologie ist in der Programmiersprache \glqq GO\grqq{} geschrieben und nutzt Funktionalitäten des
Linux Kernels, wie cgroups und namespaces.
Namespaces ermöglichen die Isolation von Prozessen in sogenannte Container, welche unabhängig voneinander arbeiten \cite{dockergetstarted}.
Diese beeinhalten alle nötigen Abhängigkeiten zur Ausführung der vordefinierten Anwendung.
Container gewinnen dadurch an Portabilität, die ein bereitstellen auf Infrastrukturen mit der Docker
Laufzeit ermöglichen.
Die Laufzeit setzt sich aus \glqq runc\grqq{} einer low-Level Laufzeit und \glqq containerd\grqq{} einer higher-Level
Laufzeit zusammen (vgl. Abbildung~\ref{fig:dockerarch}).
Runc dient als Schnittstelle zum Betriebssystem und startet und stoppt Container.
Containerd verwaltet die Lebenszyklen eines Container, ziehen von Images, erstellen von Netzwerken und
Verwaltung von runc.
Die Allgemeine Aufgabe des Docker Daemons ist es eine vereinfachte Schnittstelle für die Abstraktion
der unterliegenden Schicht zu gewährleisten, wie zum Beispiel dem verwalten von Images, Volumes und Netzwerken \cite{dockerdeep}.
Auf die Orchestrierung mit Swarm wird nicht weiter eingeganen, da sie zum Verständnis nicht nötig ist.

\begin{figure}
    \centering
    \includegraphics[width=0.5\columnwidth]{images/DockerArch.png}
    \caption{Docker Architektur \protect\cite{dockerdeep}}
    \label{fig:dockerarch}
\end{figure}

\subsection{Images und Container}
Ein Docker Image ist ein Objekt das alle Abhängigkeiten wie Quellcode, Bibliotheken und Betriebssystem
Funktionen für eine Anwendung beeinhaltet. 

\subsubsection{Registries}
Das beziehen von Images erfolgt über sogenannte \glqq Image Registries\grqq{}.
Bei Docker ist dies standardmäßig \url{https://hub.docker.com} und das eigene Lokale Registry. 
Es ist auch möglich eigene zu hosten oder die von Drittanbieter zu nutzen.

\subsubsection{Schichten}
Docker Images bestehen aus mehreren Schichten, jede davon abhängig von der Schicht unter ihr und
erkennbar durch IDs in Form von SHA256 Hashes (vgl. Abbildung~\ref{fig:dockerlayer}).
Docker kann dadurch beim bauen oder updaten von neuen Images vorhandene Schichten erneut verwenden. 
Die feste Reihenfolge ermöglicht eine ressourceneffiziente Verwaltung von Builds,
indem man oft wechselnde Schichten oben platziert. 
Die Leistung beim erstellen und zusammenführen von Schichtem hängt vom Dateisystem des Hostsystems ab.
Eine Schicht kann aus mehreren Dateien bestehen
und einzelne Dateien aus der Unterliegenden Schicht mit einer neuen ersetzen.

\begin{figure}
    \centering
    \includegraphics[width=0.5\columnwidth]{images/Image-Layer.png}
    \caption{Image Layers \protect}
    \label{fig:dockerlayer}
\end{figure}

%copy-on-write (CoW) strategy???

%nicht alle dockerfile anweisungen erstellen layer ENV expose usw.
Das starten eines Containers fügt auf die bereits bestehenden Schichten einen \glqq Thin R/W layer\grqq{} 
oder auch \glqq Container layer\grqq{} genannt hinzu, dieser gewährt Schreib- und Leserechte bei Laufzeit des Prozesses. 
Jeder dieser Container hat somit einen individuellen Zustand, der unähnlich vom abstammendem Image ist.
Bei Löschung des Containers verschwindet auch die dazu gewonne Schicht.
Das entfernen eines Images ist durch die Konzeption des Schichtensystem erst möglich, wenn alle darauf
basierenden Container gelöscht sind \cite{dockerstoragedriver}.

\subsubsection{Dockerfile}
Zur Erstellung eines Docker Images wird ein Dockerfile benötigt, dies beeinhaltet alle Anweisungen
zum Aufbau der einzelnen Schichten. 
Diese Aufrufe erstellen die Schichten eines Images \cite{dockerbestpracticedockerfile}.
\begin{itemize}
    \item \textbf{FROM} erstellen einer Schicht von einem base-image. 
    \item \textbf{COPY} hinzufügen von Dateien aus dem derzeitigen Verzeichnis. 
    \item \textbf{RUN} bauen der Anwendung mit make. 
\end{itemize}
Diese hingegen fügen nur Metadaten hinzu \cite{dockerbestpracticedockerfile}.
\begin{itemize}
    \item \textbf{EXPOSE} informiert Docker an welchem Port der Container innerhalb seines Netzwerks lauscht.
    \item \textbf{ENTRYPOINT} ermöglicht es einen Container als ausführbare Datei zu starten.
    \item \textbf{CMD} Befehl beim ausführen des Containers.
\end{itemize}


\subsection{Containervirtualisierung}
Aus dem Wissen des letzten Abschnitts lässt sich Schlussfolgern, dass
ein Container eine laufende Instanz eines Images ist.
Vergleichbar ist dieses Konzept mit dem einer virtuellen Maschine (VM).
Denn Images ermöglichen ähnlich wie VM templates, die Erstellung von mehreren Instanzen durch eine Vorkonfiguration.
Mit dem großen Unterschied, dass die Einrichtung von VMs müheseliger ist und weitaus mehr Ressourcen
beansprucht, da sie ein ganzes Betriebssystem ausführt \cite{hypervisorcontainer}. Containertechnologien bauen hingegen nur auf 
bestimmte Funktionalitäten des Kernels auf und sparen damit an Rechenleistung (vgl. Abbildung~\ref{fig:containervm}).

Durch die Vorteile eines geteilten Kernels und dessen Betriebssystem abhängigkeiten,
erzielen Virtualisierungen basierend auf Container eine höhere Anzahl an 
virtuellen Instanzen. Images sind auch um einiges kleiner als hypervisor-basierende
Ansätze \cite{hypervisorcontainer}.

\begin{figure}
    \centering
    \includegraphics[width=1.0\columnwidth]{images/Container-VM.png}
    \caption{Virtualisierungsmöglichkeiten \protect}
    \label{fig:containervm}
\end{figure}

Die Einsparung von Ressourcen und dem einfachen bereitstellen auf Hostsysteme,
prädestinieren containerisierte Anwendungen für die Verwendung von Microservices
auf Container Plattformen wie Kubernetes.


\section{Kubernetes}

Dieser Abschnitt befasst sich zunächst mit den einzelnen Komponenten der Kubernetes Architektur.
Hinleitend werden spezielle Themen wie k3s, Hybrid Cloud und Rancher näher erläutert.
Kubernetes ermöglicht die Orchestrierung von containerisierten Arbeitslasten
und Services. Seit 2014 hat Google das Open-Source-Projekt zur Verfügung
gestellt und baut auf 15 Jahre Erfahrungen mit Produktions-Workloads \cite{kubernetes}.

\subsubsection{Namensgebung}
\glqq Der Name Kubernetes stammt aus dem Griechischen, bedeutet Steuermann oder Pilot, [...]
K8s ist eine Abkürzung, die durch Ersetzen der 8 Buchstaben \glqq ubernete\grqq{} mit \glqq 8\grqq{} abgeleitet wird\grqq{} \cite{kubernetes}.

\subsection{Cluster}
Die Zusammensetzung der beschriebenen Kubernetes Komponenten ergeben ein Kubernetes Cluster (vgl. Abbildung~\ref{fig:cluster}).


\begin{figure}
    \centering
    \includegraphics[width=1.0\columnwidth]{images/components-of-kubernetes.png}
    \caption{Komponenten eines Kubernetes Cluster \cite{kubernetesnodekomponenten}}
    \label{fig:cluster}
\end{figure}

\subsubsection{Control Plane}
Control Planes\footnote{Seit Kubernetes v1.20, ist die korrekte Bezeichnung für die Master Node Control Plane \cite{Kuberneteschangemaster}} sind für die Steuerungsebene des Clusters zuständig,
dabei entscheidet und reagiert dieser auf globaler Ebene auf eintreffende Clustereignisse \cite{kuberneteskomponenten}.
Die nächsten Unterabschnitte beschreiben diese näher:

\subsubsection{\textit{API Server}}
Der API Server operiert über REST und bietet eine Schnittstelle zu Diensten
inner- und außerhalb der Master-Komponenten \cite{kuberneteskomponenten}.

\subsubsection{\textit{etcd}}
etcd ist der primäre Datenspeicher von Kubernetes und sichert alle Zustände eines
Cluster \cite{kuberneteskomponenten}.

\subsubsection{\textit{Scheduler}}
Der Scheduler ist zuständig für die Verteilung und Ausführung von Pods auf Nodes \cite{kuberneteskomponenten}.

\subsubsection{\textit{Controller Manager}}
Der Controller Manager reagiert auf Ausfälle von Nodes, erhält die korrekte
Anzahl von Replikationen eines Pods und verbindet Services miteinander \cite{kuberneteskomponenten}.

\subsubsection{Node}
Eine Node\footnote{Um den Sprachfluss zu wahren wird der englische Begriff Node, als Kubernetes Ressourcenobjekt nicht übersetzt. 
Die Übersetzung Knoten findet als Hardwareinstanz statt.} 
ist eine Hardware Einheit, die je nach Kubernetes Einrichtung eineVM, physische Maschine oder 
eine Instanz in einer privaten oder öffentlichen Cloud darstellen kann \cite{kubernetesnodes}.
Dieser umfasst folgende Komponenten:

\subsubsection{\textit{Container Laufzeit}}
Der Abschnitt \hyperref[Docker]{2.1} beschreibt bereits alles zu diesem Thema.
Ergänzend dazu die Information, dass seit 2020 containerd als Auslaufmodell für die unterliegende Container Laufzeit, 
für die Kubernetes Versionen nach v1.20 auslaufen. Dies beeinträchtigt die spätere Implementierung dieser Arbeit aber nicht,
da k3s containerd als standard Laufzeit verwendet wird.

\subsubsection{\textit{Kubelet}}
Kubelet fungiert als \glqq node agent\grqq{} und registriert die Node mit dem
API-Server eines Clusters, dabei stellt es sicher das Container innerhalb eines Pods
funktionieren \cite{kubernetesnodes}.

\subsubsection{\textit{Kube-Proxy}}
Ein Kube-Proxy ist ein Netzwerk Proxy und verwaltet die Netzwerkrechte auf Nodes.
Diese erlauben die Kommunikation zwischen Pods inner- und außerhalb des Clusters \cite{kubernetesnodes}.

\subsection{Pods}
Ein Pod stellt die kleinste Einheit eines Kubernetes Clusters dar und ist eine Gruppe von mindestens einem Container.
Dieser erlaubt die gemeinsame Nutzung von Speicher- und Netzwerkressourcen mit Anweisungen zur Ausführung der Container \cite{kubernetespods}.

\subsection{Deployment}
Ein Deployment ist ein Ressourcenobjekt, dass mit einem Deployment Controller den gewünschten Zustand einer Anwendung aufrechterhält.
Diese Spezifikationen sind in Form von YAML-Dateien definiert (vgl. Beispiel~\ref{lst:deployment}).
Desweiteren eine kurze Aufschlüsselung der einzelnen Instruktionen \cite{kubernetesobjects}.
\begin{itemize}
    \item \textbf{apiVersion}: definiert die einzelnen workload API Untergruppen und die Version.
    \item \textbf{kind}: bestimmt das zu erstellende Kubernetes Objekt.
    \item \textbf{metadata}: deklariert einzigartige Bestimmungsmerkmale.
    \item \textbf{spec}: gewünschte Ausgangszustand des Objekts.
\end{itemize}

\begin{lstlisting}[caption={deployment.yaml \cite{kubernetesdeployment} },captionpos=b,label={lst:deployment},language=yaml]
    apiVersion: apps/v1
    kind: Deployment
    metadata:
      name: nginx-deployment
      labels:
        app: nginx
    spec:
      replicas: 3
      selector:
        matchLabels:
          app: nginx
        spec:
          containers:
          - name: nginx
            image: nginx:1.14.2
            ports:
            - containerPort: 80
    \end{lstlisting}

\subsubsection{Deployments und Pods}
Das Einbinden von Pods in Deployments ermöglicht Kubernetes das beziehen von 
wertvollen Metadaten für die Verwaltung von Skalierbarkeit,
Rollouts, Rollbacks und Selbstheilungsprozesse \cite{kubernetesnigeldeployments}. Der höhere Grad an Abstraktion
dient auch zur Aufteilung von Microservice Stacks, zum Beispiel dem aufteilen
von Frontend und Backend Pods in eigene Deployment Zyklen.

\subsection{Service}
Ein Service ist für die Zuweisung von Netzwerkdiensten einer logischen Gruppe Pods zuständig.
Services dienen als Abstraktion von Pods und ermöglichen die Replizierung und Entfernung
von Pods ohne Beeinträchtigung der laufenden Anwendung \cite{kubernetesservice}.

Pods beanspruchen Netzwerkressourcen, wie IP-Adresse und DNS-Name 
innerhalb ihres Clusters. Der Ausfall oder die Zerstörung eines Pods führt zu Beeinträchtigung der Kommunikation
zwischen Anwendungen. Services können dies präventiv verhindern, indem sie mit
selector und labeler eine Kommunikation zwischen zwei Kubernetes Objekten etablieren.
Das Beispiel zeigt eine solche Konfiguration (vgl. Beispiel~\ref{lst:service}). 
Die einzelnen Spezifikationen werden folgendermaßen definiert \cite{kubernetesservice}:

\begin{itemize}
  \item \textbf{selector}: definiert die Abbildung auf ein Label.
  \item \textbf{app}: führt den Service für Pods mit dem vorgegebenen Label aus.
  \item \textbf{ports}: Netzkonfiguration zwischen Service und Pod.
  \item \textbf{targetPort}: Port auf dem die Anwendung im Pod lauscht.
  \item \textbf{port}: Port auf dem der Service lauscht.
\end{itemize}

\begin{lstlisting}[caption={zum deutlicheren Verständnis mit dem deployment.yaml, leicht abgewandelte service.yaml \cite{kubernetesservice} },captionpos=b,label={lst:service},language=yaml]
    apiVersion: v1
    kind: Service
    metadata:
      name: my-service
    spec:
      selector:
        app: nginx
      ports:
        - protocol: TCP
          port: 80
          targetPort: 9376
    \end{lstlisting}

Bei der Erstellung eines Services wird ein REST Objekt erstellt, dass mithilfe eines Controller kontinuierlich 
nach Pods mit dem passenden Selector sucht, welcher jegliche Updates als POST-Anfragen schickt.


\subsection{Ingress}

Ein Ingress ist ein Kubernetes Ressourcenobjekt, dass die Verfügbarkeit von internen Services auf öffentliche Endpunkte ermöglicht.
Diese Routen werden mittels HTTP oder HTTPS freigegeben und können in Form einer URL verwendet werden \cite{kubernetesingress}.
Die Anforderung für die Implementierung eines Ingress ist der Ingress-Controller, eine Vielzahl an Optionen dafür wird in der 
Dokumentation aufgelistet \cite{kubernetesingresscontroller}. Für die Realisierung des Prototyps kommt ein NGINX Ingress Controller in Einsatz, weshalb
dieser näher erläutert wird.

\subsubsection{NGINX Ingress Controller}
Der Ingress Controller ist für die Umsetzung einer vorgegebenen Objektspezifikation zuständig \cite{kubernetesingress}.
Die übliche Verwendung eines Controllers beeinhaltet die Lastenverteilung durch weiterleiten des Datenverkehrs an Services. 
Diese Kommunikation findet, wie auch bei dem NGINX Ingress Controller \cite{kubernetesingresscontrollerlayer} in der Anwendungsschicht des OSI-Schichtenmodells statt und ermöglicht, dadurch die 
Lastenverteilung von öffentlichen Endpunkten zu internen Pods in einem Cluster \cite{kubernetesingressibm}.
Wie in alle anderen Kubernetes Objekten auch werden vordefinierte Aufgaben des Ingress Controller durch YAML-Dateien abgebildet (vgl. Beispiel~\ref{lst:ingress}).
Im folgenden wichtige Optionen die etwas genauer erklärt werden:

\begin{itemize}
  \item \textbf{ingressClassName}: definiert den Ingress Controller.
  \item \textbf{rules}: die Zusammsetzung der einzelnen HTTP Regeln.
  \item \textbf{host}: definiert das Ziel des eintreffenden Datenverkehrs.
  \item \textbf{paths}: gibt die Endpunkte des verbundenen Service an.
  \item \textbf{backend}: leitet die Anfragen an den Service mit der richtigen Port Zuweisung weiter.
\end{itemize}

\begin{lstlisting}[caption={ingress.yaml \cite{kubernetesingress} },captionpos=b,label={lst:ingress},language=yaml]
  apiVersion: networking.k8s.io/v1
  kind: Ingress
  metadata:
    name: minimal-ingress
    annotations:
      nginx.ingress.kubernetes.io/rewrite-target: /
  spec:
    ingressClassName: nginx
    rules:
    - http:
        paths:
        - path: /testpath
          pathType: Prefix
          backend:
            service:
              name: test
              port:
                number: 80
  \end{lstlisting}

\subsection{Lightweight Kubernetes} \label{k3s}
Ligthweight Kubernetes auch K3s genannt ist eine Kubernetes Distribution von dem Unternehmen
Rancher. Der größte Unterschied der Distribution ist die einnehmende größe 
auf Hostsystemen mit einer einzelnen Binärdatei von nur 40MB ist auch Platz auf kleineren Geräten.
Der hauptsächliche Verwendungszweck von k3s liegt in IoT und Edge-Devices, da unwichtige Kubernetes Inhalte entfernt wurden. \cite{k3s}.
Trotz dieser Reduzierung, bleiben die Kernfunktionalitäten von Kubernetes erhalten und werden 
so weit wie möglich parallel auf dem neusten Stand gehalten \cite{k3sgit}.

\subsubsection{Besonderheiten}
Die Abbildung \ref{fig:k3sarchitektur} zeigt die Architektur von k3s auf. Das Kubernetes äquivalent zur Control Plane und Node
sind Server und Agent. Eine Besonderheit dessen ist, das Server parallel einen Agent Prozess auf dem selben Knoten starten und
somit Arbeitslasten mithilfe von Kubelet ausführen. Weiterhin wird im Gegensatz zu k8s containerd weiterhin unterstüzt und
kommt vorinstalliert mit Kubelet. Zwei weitere Besonderheiten werden näher erläutert:

\textbf{Kine}
das Akronym steht für 'Kine is not etcd' und ist eine Abstraktionsschicht von etcd, welche sqlite, Postgress, Mysql und dqlite übersetzt \cite{k3sgit}.
Dadurch kann der Backend Speicher des Clusters durch die oben genannten Datenbanksysteme ersetzt werden.

\textbf{Flannel}
ist ein überlagerndes Netzwerkmodell in k3s und ermöglicht IPv4 Netzwerke innerhalb eines Clusters mit mehreren Knoten.
Eine Binärdatei startet Agents auf Hostssystemen und alloziert Subnetze in einem vorkonfigurierten Adressraum.
Das Modell ist dabei für die Übertragungsart des Datenverkehrs zwischen unterschiedlichen Knotenpunkten zuständig.
Die Speicherung der Netzwerkkonfiguration erfolgt über etcd oder der Kubernetes API \cite{flannel}.


\begin{figure}
  \centering
  \includegraphics[width=1.0\columnwidth]{images/k3s-architecture.jpeg}
  \caption{k3s Architektur \cite{k3sarch}}
  \label{fig:k3sarchitektur}
\end{figure}

\subsection{Rancher}
In diesem Abschnitt wird die Open-Source Lösung Rancher von dem gleichnamigen Unternehmen zur Orchestierungs von Kubernetes Clustern näher behandelt.
Diese ermöglicht, das Verwalten von Kubernetes Clustern auf der eigenen Infrastruktur, sowohl vor Ort als auch in der Cloud.
Die Bereitstellung von Clustern mittels Rancher ist Cloud-Anbieter unabhängig,
weshalb Cluster in der Praxis mit der selben Rancher Instanz auf AWS, Azure oder anderen Cloud-Anbietern betreut werden können \cite{rancher}.

Die Rancher Benutzeroberfläche vereinfacht das steuern von Arbeitslasten, auf einer zentralen administrativen Instanz, welche gleichzeitig Authentifizierung und Rechteverteilung von Benutzern anbietet.
Das grundsätzliche verwalten von Arbeitslasten verlangt kein tiefgründiges Wissen bezüglich Kubernetes Konzepte. 
Die mitgelieferten Tools ermöglichen die Auslieferung und Verbindung von Kubernetes Objekten und abstrahieren die Komplexität, die für die Betreuung eines solchen Systems notwendig sind.
Für komplexere Konfigurationen, kann über die Oberfläche ein Terminal mit Kubectl aufgerufen werden \cite{rancher,AzureKubernetesService}.

\begin{figure}
  \centering
  \includegraphics[width=1.0\columnwidth]{images/RancherArchitekturClusterController.png}
  \caption{Rancher Server Kommunikation mit einem downstream k3s Cluster, überarbeitete Abbildung von \cite{rancherArchitecture}. (Für die spätere Implementierung nachgebildet)}
  \label{fig:rancherarchitektur}
\end{figure}

Die Abbildung \ref*{fig:rancherarchitektur} zeigt den Vorgang von Zwei Benutzern, 
die auf ein von Rancher verwalteten downstream Cluster\footnote{Die offiziele Bezeichnung für ein Kubernetes Cluster unter Rancher ist \textbf{downstream Cluster} \cite{rancherArchitectureRecommendations}} zugreifen.
Die nachfolgende Beschreibung aus der Dokumentation gibt die einzelnen Schritte mit der in der Abbildung nummerierten Posten wieder \cite{rancherArchitecture}.
\begin{enumerate}
\item Zuerst authentifiziert sich Bob mit seinen Benutzerdaten bei dem Authentifizierungs-Proxy für seine Rancher Instanz.
Dieser Proxy leitet den Aufruf mit der ausgewählten downstream-Cluster Instanz weiter und führt diese aus.
Dafür wird vor dem weiterleiten des Aufrufs, der angemessene Kubernetes Impersonation Header gesetzt, 
welcher sich als Service Account der Rancher Instanz ausgibt und je nach Benutzerrecht reagiert.   
\item Die Übertragung des Aufrufs erfolgt über einen Cluster-Controller auf dem Rancher Server
und dem parallel laufendem Cluster-Agent auf dem downstream-Cluster. Der Controller ist für die Überwachung, Veränderung
und Konfiguration von Zuständen auf dem laufendem Cluster zuständig. 
\item Wenn der Cluster-Agent nicht erreichbar ist,
werden die Aufrufe an den Node-Agent\footnote{Ein Rancher DaemonSet zur Interaktion mit Nodes, 
nicht zu verwechseln mit dem k3s-Agent \cite{rancherAgents}. } überreicht, welcher standardmäßig auf jedem downstream-Cluster läuft.
\item Zuletzt hat auch die Benutzerin Alice, die Möglichkeit sich über einen autorisierten Cluster Endpunkt zu verbinden.
Denn jeder downstream-Cluster verfügt, über eine Kubeconfig, welche den Zugang ohne Authentifizierungs-Proxy erlaubt.
Durch den Microservice kube-api-auth wird eine Kommunikation, über einen Web-Haken realisiert, der die Verbindung
zwischen Alice Rechner ermöglicht. 
Dies ermöglicht die Verwendung von Befehlszeilentools, wie Kubectl und Helm.
\end{enumerate}

%\section{Distributed Clouds}
%\subsection{Edge Computing}
\subsection{Hybrid Cloud}
Eine hybride Cloud erlaubt die Verwendung von Hardware in der Cloud als auch vor Ort
in der selben Infrastruktur zu betreiben.

\section{Microservice}
\subsection{Aufbau}
\subsection{Entwicklung}
\subsection{Dezentrale Datenmanagement}


%\subsubsection{Ein Unterabschnitt}

  \chapter{Anforderungen}

Das vorherige Kapitel beschreibte die nötigen Technologien für die Realisierung und Ausführung einer Microservice Architektur.
Dieses Kapitel beschäftigt sich nun mit der Analyse 

\section{Analyse}
\subsection{Anwendungsszenario}
\section{Spezifikation}
\subsection{Use-Case}
  \chapter{Konzeption}
Das folgende Kapitel beschreibt die Konzeptionellen 
\section{Herausforderungen}
\section{KI-Anwendungsfall}
\subsection{Gesichtserkennung}
  \chapter{Lösungskonzept}\label{loesungskonzept}
Im Fokus des fünften Kapitels steht die Konzeption einer Anwendung im Microservice-Architektur-Stil.

\section{Design Entscheidungen}

Der folgende Abschnitt behandelt die gewählten Technologien für das Lösungskonzept der Microservice-Architektur.

\subsection{Backend}

\subsubsection{Flask}
Für die Entwicklung der Webanwendung in Python wird das Microframework Flask verwendet.
Dieses beinhaltet nur die wesentlichen Funktionalitäten für die Webentwicklung.
Dafür bietet das Framework eine hohe Flexibilität, da die nötigen Bibliotheken vom Entwickler gewählt werden können \cite{flaskdocu} und
es vereinfacht die Erstellung von \acs{api}s durch Blueprints \cite[S.11]{restfulpython}.

Blueprints sind ein Konzept von Flask, welche die Aufteilung von Komponenten einer Webanwendung ermöglichen.
Diese Komponenten können in Form von Routen unterschiedliche Endpunkte mit einer View ausgeben \cite{flaskdocu}.
Dadurch werden die Funktionalitäten der Webanwendung in Endpunkten strukturiert. 
Dies ist ideal für die Ausführung von losen Diensten, die über Endpunkte kommunizieren.

Weiterhin ermöglichen Flask-Abhängigkeiten, wie die Template-Engine Jinja das Rendern von HTML-Templates mit Daten aus der Flask-Anwendung.
Und das Bereitstellen einer standartisierten Schnittstelle \ac{wsgi} über Werkzeugkasten.
Diese ermöglicht die Verwendung der meisten Webserver \cite{flaskdocu}.


\subsubsection{Gunicorn}
Gunicorn ist ein \acs{wsgi}-\acs{http}-Server für Unix, welcher mit den meisten Webframeworks kompatibel ist.
Das Gunicorn-Modell teilt einen Master-Prozess in mehrere Worker-Prozesse auf.
Der Master-Prozess ist lediglich eine Schleife für die bestehenden Worker-Prozesse
und ist bei einem Ausfall für den Neustart zuständig.
Die Worker-Prozesse sind für die Verarbeitung von eingehenden Anfragen zuständig.
Diese teilen sich in folgende Worker-Klassen auf. 
Sync-Workers bearbeiten Anfragen jeweils einzeln und unterstützten keine persistente Verbindung.
Async-Workers basieren auf Greenlets und unterstützten mithilfe von Gevent asynchrone Koroutinen \cite{gunciorndocs}. 

\subsubsection{OpenCV}
OpenCV ist eine Open-Source-Computer-Vision-Bibliothek\footnote{Computer-Vision bezeichnet die Transformation von visuellen Daten in eine abgewandelte Form, die zu Beantwortung einer Fragestellung dienen kann.}, die zur Vearbeitung von Bildern verwendet wird \cite{opencvintro}.
Da ein Video nur eine Serie von Bildern ist, können die Techniken der Bildverarbeitung auch hier genutzt werden \cite{ansari_core_2020}.
Die Bibliothek beinhaltet eine Vielzahl an Algorithmen mit Bezug zu Computer Vision oder Machine Learning.
Diese untersützten auch die Verwendung von \acp{gpu} die auf den Programmierschnittstellen \ac{cuda} oder \ac{opencl} basieren \cite{opencvpython}.
Die Anwendung wird mit der Python-Version der Bibliothek entwickelt, um die Implementierung in die Python-Webanwendung zu vereinfachen.



\subsection{Frontend}

\subsubsection{\ac{html} und JavaScript}
Die Umsetzung der Benutzerobfläche erfolgt mit der Auszeichnungssprache \acs{html}5 in Kombination
mit der Skriptsprache JavaScript, um Interaktion mit dem Anwender zu ermöglichen.
Für die erleichterte Gestaltung der Website wird das Frontend-\acs{css}-Framework Bootstrap in der Version 5.0 genutzt.

\subsection{Kommunikation}

\subsubsection{Socket.IO}
Die bidirektionale und ereignisbasierte Echtzeitkommunikation zwischen den Diensten wird mithilfe der Bibliothek Socket.IO realisiert.
Die Bibliothek unterstützt mehrere Programmiersprachen für Server- und Client-Implementierungen, welche von der Community gewartet werden.
Eine Kommunikation zwischen Server und Client erfolgt über WebSockets. 
Wenn dies nicht möglich ist, wird auf die ressourcenintensivere \cite{httpwebsocket} Alternative \acs{http}-long-polling zurückgegriffen \cite{socketio}.
Für die Kommunikation der Webanwendung wird die Server Implementierung von Python-Socketio genutzt \cite{python-socketio}.
Die Implementierung der Anwendungslogik erfolgt über die Plattformunabhängige JavaScript-Bibliothek Socket.IO.


\subsubsection{\ac{rest}}
\acs{rest} ist eine auf Ressourcen basierende Architektur für verteilte Systeme.
Diese Ressourcen werden über eine einheitliche Schnittstelle basierend auf 
\acs{http} Methoden zugänglich gemacht.
Dabei ist jede Ressource über eine URL erreichbar.
REST erlaubt dabei Ressourcen in verschiedene Datentypen zu repräsentieren, wie Text, XML, JSON etc.
Die CRUD-Funktionen (create, read, update und delete) werden über die HTTP-Methoden GET, POST, PUT und DELETE realisiert \cite{fundamentalsRestfulAPI}.
\subsection{Datenbank}

\subsubsection{MongoDB}

MongoDB ist eine dokumentenorientierte Datenbank, bei der Daten nicht in einer Tabelle, 
sondern in Dokumenten gespeichert werden.
Sie zählt damit zu den NoSQL-Datenbanken.
Die Dokumentenorientierung ermöglicht die Darstellung von komplexen hierarchischen Beziehungen mit einem einzigen Eintrag.
Dokumente sind nach einer Key-Value-Struktur aufgebaut und besitzen kein vorgeschriebenes Schema zur Erstellung von Einträgen \cite{mongodbdefinitive}.

\subsection{Versionsverwaltungssystem}
\subsubsection{GitHub}
Das verwendete Versionsverwaltungssystem für die Entwicklung der Microservices ist GitHub.
Dieses basiert auf git und fokusiert sich auf Open-Source-Software und bietet gleichzeitig Enterprise-Support für Unternehmen \cite{githubpricing}.
Die Krones AG hat die Möglichkeit GitHub für zukünftige Entwicklungsprozesse von Microservices zu verwenden.

\subsubsection{DockerHub}
Wie in Abschnitt \ref{Docker} beschrieben ist die Standard-Registry für Docker-Images DockerHub.
Deshalb werden für die Bereistellung der Docker-Images die kostenfreien und öffentlichen Repositories verwendet.
Die Entwicklung der Dienste erfolgt in getrennten Repositories.

\section{Entwicklung}
In dem folgenden Abschnitt werden die Entwicklungsschritte der Microservice-Anwen\-dungen und der Ressourcenobjekte für das Bereitstellen mit Helm näher erläutert.


\subsection{Microservice-Entwicklung}
Die Abbildung \ref{fig:DevWorkflowDocker} zeigt den Arbeitsablauf der Service-Entwicklung auf.
Zuerst wird die Funktionalität des Dienstes realisiert und dann mithilfe eines Dockerfiles ein Docker-Image erstellt.
Dafür wird ein Base-Image entweder aus einem Docker-Registry, wie DockerHub, oder aus dem lokalen Registry benötigt.
Um den Container zu starten, kann ein Docker-Befehl auf dem Hostssystem ausgeführt werden.
Die Nutzung von Tools wie Docker-Compose erlauben das Starten mehrerer Container mithilfe von Konfigurationsdateien in Form von YAML-Dateien.
Nach dem Ausführen der Container können diese getestet werden.
Abschließend kann der Entwicklungsablauf fortgeführt werden oder ein Release für das Versionsverwaltungssystem und das Container-Repository erstellt werden.


\begin{figure}[!htb]
    \centering
    \includegraphics[width=1.0\columnwidth]{images/DevWorkflowDocker.png}
    \caption{Microservice-Entwicklung in Anlehnung an \cite{dockerappsworkflow}}
    \label{fig:DevWorkflowDocker}
\end{figure}


\subsection{Helm-Chart-Entwicklung}

Die Abbildung \ref{fig:DevWorkflowKubernetes} beschreibt den Vorgang bei der Entwicklung von Helm-Charts für Kubernetes.
Als Erstes wird ein Helm-Chart entwickelt.
Der Kommandozeilenbefehl \textit{helm lint} überprüft den vorgegebenen Pfad zum Chart und führt eine Serie von Tests zur Validierung durch.
Danach kann dieser auf einem Kubernetes-Cluster installiert werden, 
wenn die Kubernetes-Ressourcenobjekte einen Docker-Container benötigen,
wird das spezifizierte Image aus dem öffentlichen DockerHub-Registry heruntergeladen.
Der Service kann jetzt mit dem Kommandozeilentool Kubectl getestet werden.
Zuletzt wird die Entwicklung am Helm-Chart fortgeführt oder das Egebnis Artefakt ins Versionsverwaltungssystem hochgeladen.


\begin{figure}[!htb]
    \centering
    \includegraphics[width=1.0\columnwidth]{images/DevWorkflowKubernetes.png}
    \caption{Kubernetes-Entwicklung in Anlehnung an \cite{dockerappsworkflow}}
    \label{fig:DevWorkflowKubernetes}
\end{figure}


\newpage
\section{Architektur}
Dieser Abschnitt befasst sich mit der Architektur der zu entwickelnden Anwendung.
Die Aufgaben der einzelnen Dienste der Microservice-Architektur wurden konzipiert und dargestellt.
Danach folgt der Vorgang der Installation der losen Dienste mit Helm auf dem Kubernetes-Cluster.

\subsection{Microservices}

Die Abbildung \ref{fig:UML-Microservices} zeigt die einzelnen Softwarekomponenten der Webanwendung, welche als Docker-Container laufen.
Das Frontend dient als visuelles Gateway für die anderen Dienste.
Dieses bietet vier Endpunkte, die für Nutzer über einen Webbrowser erreichbar sind.
Der Home-Endpunkt ermöglicht den Login oder Logout eines Nutzers über die REST-API des Authentication-Dienstes.
Register erlaubt die Registrierung eines Nutzers in der Datenbank.
Train und Facelogin senden Bilder an den Facerecognition-Dienst, dies geschieht mit dem Kommunikationsprotokoll SocketIO.
Damit wird das Modell zur Gesichtserkennung trainiert und ermöglicht die spätere Zwei-Faktor-Authentifizierung mittels Login per Gesichtserkennung.


\begin{figure}[!htb]
    \centering
    \includegraphics[width=1.0\columnwidth]{images/UML-Microservices-Diagramm.png}
    \caption{Lokale Microservice Entwicklung}
    \label{fig:UML-Microservices}
\end{figure}




\subsection{Helm-Installation}

Die einzelnen Dienste der Webanwendung werden mithilfe von einem Helm-Chart gleichzeitig auf ein Kubernetes-Cluster installiert.
Dabei hat jeder Dienst ein eigenes Verzeichnis mit den notwendigen Kubernetes-Ressourcenobjekten.
Der Zugang erfolgt über eine Kubeconfig die den Zugang zum Kubernetes-Cluster ermöglicht.
Bei erfolgreichem Zugang kann mit einem Befehl im Verzeichnis die Microservices installiert werden.
Das Kubernetes-Cluster bezieht dann die benötigten Docker-Images aus den angegebenen Docker-Repositories.
Die erfolgreiche Bereistellung der Container auf dem Cluster ist dann unabhängig von der Helm-Installation, wenn die Images nicht von den angegebenen Repositories bezogen werden können. 
Die Bereistellung und Auslieferung der Kubernetes-Ressourcenobjekte ist trotzdem erfolgreich und gibt auf dem Kubernetes-Cluster lediglich Fehlermeldungen, bei der versuchten Ausführung der Container in einem Pod an.

\begin{figure}[!htb]
    \centering
    \includegraphics[width=1.0\columnwidth]{images/BPMNHelm.png}
    \caption{BPNM Modell - Helm--Installation der Microservices}
    \label{fig:HelmInstallation}
\end{figure}






  \chapter{Umsetzung des Lösungskonzepts}

Das folgende Kapitel beschreibt die Vorgehensweisen der Implementierung.
Angefangen mit der Konfiguration und Einrichtung der Knotenpunkte für das Kubernetes-Cluster.


\section{Konfiguration und Einrichtung}

In diesem Abschnitt geht es um die Einrichtung der Kubernetes Infrastruktur.
Zuerst die Einrichtung der einzelnen virtuellen privaten Server in Vultr, die als Knotenpunkte in unserem Kubernetes Cluster funktonieren.
Danach die Installation der Infrastruktur mit k3s.
Zunächst wird eine Domain für den späteren Einsatz der Microservices konfiguriert.
Abschließend erfolgt die Bereitstellung von Zertifikaten für die Domain.

\subsubsection{Virtueller privater Server}

Durch die Einschränkungen, beschrieben in Abschnitt \ref{Einschraenkungen},
werden für die Installation der Kubernetes Plattform virtuelle private Server (VPS) verwendet.
Ein VPS ist eine virtuelle Maschine, die von Drittanbietern wie Internet-Hosting-Diensten, als Dienst verkauft wird.
Dies ermöglicht das Mieten von Hardware.
Die Server dienen als Knotenpunkte für die spätere Kubernetes Installation.
Es werden ingesamt drei VPS-Instanzen gemietet auf denen das Betriebssystem SLE-Micro Enterprise 5.1 bereitgestellt und auf den Serverinstanzen installiert.

\subsubsection{Domain}
Der Zugang zur Webanwendung wird mithilfe einer öffentlichen Domain ermöglicht.
Der DNS-Eintrag einer Domain ist für die Adressierung zuständig.
Durch die Veränderung des A-Records leiten alle Anfragen der Domain auf eine IPv4-Adresse um \cite{LearningCoreDNS}.
Die IPv4 Adresse ist in diesem Fall der Cluster Master der späteren k3s-Installation.

\subsection{KubeVision}
Dieser Abschnitt behandelt die einzelnen Softwarekomponenten der Microservice-Anwendung KubeVision.
Die Webanwendung ist in drei verschiedene Dienste unterteilt.
Erstens der Benutzeroberfläche für die Interaktion mit dem Benutzer.
Zweitens dem Authentifizierungsdienst, der für die Registrierung und Anmeldung zuständig ist.
Drittens der Backend-Dienst, welcher die Authentifizierung per Gesichtserkennung ermöglicht.

\subsection{Frontend-Service}
Frontend-Service ist die Benutzeroberfläche zur Interaktion mit dem Benutzer.
Der Dienst ist in mehrere Blueprints mit eigenen Endpunkten aufgebaut.
Jeder dieser Endpunkte gibt einen URL-Pfad für die Interaktion mit dem Frontend-Service oder einem anderen Dienst an.
Bei Aufruf eines Endpunkts wird eine view aufgerufen und mithilfe der Template Engine Jinja2 eine spezifische HTML-Datei aus dem templates-Verzeichnis ausgegeben.
Diese spezifische Datei ist ein HTML Code-Block und wird in die Main-View gesetzt.

Es gibt Zwei Blueprints einer im Verzeichnis home, welcher die Funktionalitäten und Endpunkte für das einloggen, registrieren und anzeigen des Profils ausgibt.
Für die Authentifizierungsmöglichkeiten wird auf die Authentication-Service-Endpunkte umgeleitet.
Das Zweite Blueprint gibt views zum Interagieren mit dem Facerecognition-Service an.
Dieser beinhaltet Logik in Form von JavaScript mit der eingebundenen Bibliothek SocketIO und ermöglicht das Senden von Bildern mithilfe einer Webcam.

Für die Kommunikation mit dem Facerecognition-Service wird die Kamera des Benutzers benötigt.
Das Modul und die enthaltene Klasse Camera.js ist für die Verwendung der Webcam zuständig.
Die Funktion navigator.mediaDevices.enumerateDevices() listet alle angeschlossenen Peripheregeräte mit Kamerafunktion auf.
Diese Geräte werden dann in einer Schleife in eine Dropdown-Liste platziert.
Der Nutzer kann danach eine spezifische Kamera auswählen.

Mit der Klasse socketio.js lässt sich die bidirektionale Kommunikation mit dem Facerecognition-Service aufbauen.
Es gibt drei unterschiedliche Events für die Kommunikation mit dem Dienst.
Stream sendet eine bestimmte Anzahl an Bildern an den Dienst und löst im Anschluss das Event traindata aus.
Dieses Event wird über den Endpunkt train ermöglicht.
Der Zweite Endpunkt facelogin ermöglicht die Kommunikation über das Event Predict.
Dieser sendet eine bestimmte Anzahl an Bildern an den Dienst und ermöglicht den Login des Nutzers.



\subsection{Authentication-Service}




\subsection{Backend-Service}


\section{Gesichtserkennung}
\subsection{Alignment}
\subsection{Training}
\subsection{Model}

\section{Dockerisierung}

Der nächste Schritt ist die Dockerisierung der losen Dienste.

\subsection{Dockerfile}

Jeder Dienst verfügt über ein eigenes Dockerfile mit Anweisungen zum Erstellen eines Docker-Images.
Um Redundanz zu vermeiden wird im folgenden das Dockerfile zum Facerecognition-Service näher erläutert (vgl. Quellcode~\ref{lst:Dockerfile}).
Dieser ist ähnlich aufgebaut wie die Dockerfiles der anderen Dienste.
Das Dockerfile befindet sich im selben Verzeichnis, wie die Code-Dateien des Dienstes.


\begin{lstlisting}[caption={Dockerfile},captionpos=b ,label={lst:Dockerfile},language=Dockerfile]
    FROM python:3.7.2-stretch
    
    WORKDIR /app
    ADD . /app
    
    RUN apt-get update
    RUN apt-get install ffmpeg libsm6 libxext6  -y
    RUN pip install --upgrade pip setuptools wheel
    RUN pip install -r requirements.txt
    
    ENV PYTHONUNBUFFERED 1
    EXPOSE 5000
    
    CMD ["gunicorn" , "-k" ,"geventwebsocket.gunicorn.workers.GeventWebSocketWorker", "-w", "3" , "--bind" , ":5000" , "run:app"]
    \end{lstlisting}

Die Basis des Docker-Images ist ein Python-Stretch-Image, welches auf dem leichtgewichtigen Betriebssystem Debian-Stretch aufbaut.
Zunächst werden die nötigen Bibliotheken zur Ausführung von OpenCV installiert.
Danach wird mit pip die notwendigen Pythonbibliotheken installiert.
In der requirements.txt stehen alle Bibliothekennamen mit der erforderlichen Version.
Die Enviornmental-Variable ermöglicht die Ausgabe des Python-Buffers im Terminal.
Die CMD Anweisung des Containers startet immer mit dem Befehl einen Gunicorn-Webserver auszuführen.
Die zusätzlichen Flags geben die Art und Anzahl der Worker-Prozesse an.
Letztendlich wird die Webanwendung mit der WSGI-Schnittstelle an den gewählten Port 5000 ausgeführt.

\subsection{Docker-Compose}



\subsection{DockerHub}

\section{Helm-Chart}
Dieser Abschnitt beschreibt die Entwicklung der Kubernetes-Ressourcenobjekte für die Bereitstellung mit dem Package-Manager Helm.
Um Redundanz zu vermeiden wird die Umsetzung der Kubernetes-Ressourcenobjekte beispielhaft am Dienst Facerecognition gezeigt.
Helm-Charts verfügen über eine YAML-Datei namens Values, welche globale Variablen für das Helm-Chart definiert.
Dadurch können die Kubernetes-Ressourcenobjekte von einer Datei aus vorkonfiguriert werden.


\subsection{Service}
Jeder Dienst verfügt über einen eigenen Kubernetes-Service.
Dieser gibt die Ports des Dienstes an.

\subsection{Ingress}

Für die Implementierung der Webanwendung wird ein Nginx-Ingress verwendet.
Dieser stellt den Endpunkt eines Services in Form einer URL dar.
Die folgenden Schritte sind zur Bereitstellung des Ingress notwendig.


\subsubsection{SSL-Verschlüsselung}
Die Verwendung der Webcam eines Benutzers ist nur in einem sicheren Kontext möglich.
Die Kommunikation zwischen einem Client und Ingress muss TLS-Verschlüsselt sein, um JavaScript Methoden wie MediaDevices.getUserMedia() auszuführen.
Dafür benötigt der Ingress-Controller ein Zertifikat und einen privaten Schlüssel.
Dieser kann automatischen mit einem Kubernetes-Issuer erstellt werden und von einem Ingress referenziert werden \cite{certmanager}.

\textbf{Issuer}: 
Das add-on Cert-Manager kommt vorinstalliert mit k3s und automatisiert die Verwaltung von Zertifikaten.
Dieser enthält die Kubernetes-Resource Issuer, welche zur Generieriung von privaten Schlüsseln dient.
ERKLÄREN WIE ACME FUNKTIONIERT!

\begin{lstlisting}[caption={issuer.yaml \cite{certmanageracme} },captionpos=b,label={lst:issuer},language=yaml]
    apiVersion: cert-manager.io/v1
    kind: Issuer
    metadata:
      name: letsencrypt-prod
    spec:
      acme:
        server: https://acme-v02.api.letsencrypt.org/directory
        privateKeySecretRef:
          name: letsencrypt-key
        solvers:
        - http01:
           ingress:
             class: nginx

\end{lstlisting}

Die Ausführung des Issuers erzeugt einen privaten Schlüssel mit der Bezeichnung letsencrypt-key und dem Kubernetes-Issuer namens letsencrypt-prod.  

\textbf{Cert}: 
Der nächste Schritt ist die Erzeugung eines Zertifikats mit dem Issuer.

\begin{lstlisting}[caption={cert.yaml \cite{certmanageracme} },captionpos=b,label={lst:cert},language=yaml]
    apiVersion: cert-manager.io/v1
    kind: Certificate
    metadata:
      name: cert-prod
    spec:
      secretName: deploy-secret
      issuerRef: 
        name: letsencrypt-prod
      dnsNames:
      - "your-domain.com"

\end{lstlisting}

Die Ausführung des Kubernetes-Cert erstellt ein signiertes Zertifikat.
Das erzeugte Secret mit der Bezeichnung deploy-secret kann von einem Ingress zur Verschlüsselung der Kommunikation verwendet werden.

\subsubsection{Nginx-Ingress}

\subsection{Deployment}



  
  % ... weitere Kapitel
 
  % Literaturverzeichnis
  \phantomsection
  \addcontentsline{toc}{chapter}{Literaturverzeichnis}

  \newpage

  \bibliography{bibliography}
  \bibliographystyle{IEEEtran}
  
  % Anhang
  \phantomsection
  \addcontentsline{toc}{chapter}{Abbildungsverzeichnis}
  \listoffigures
  \newpage

  \phantomsection
  \addcontentsline{toc}{chapter}{Tabellenverzeichnis}
  \listoftables
  \newpage

  
  %\include{anhang}
\end{document}    