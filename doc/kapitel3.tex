\chapter{Analyse}

Das vorherige Kapitel beschreibte die nötigen Technologien für die Realisierung und Ausführung einer Microservice Architektur.
Dieses Kapitel beschäftigt sich nun mit der Analyse. 

\section{Anforderungen}

In diesem Abschnitt werden wir die Anforderungen für die 


\subsection{Proof Of Concept}
Die Krones AG entwickelt ein Konzept für eine wartungsfreie Infrastruktur mit einem continuous delivery System, um durch Microservices in der Cloud die Produktionsanlage Standortübergreifend zu modernisieren. 
Dessen Umsetzbarkeit wird in einem Proof of Concept unter Beweis gestellt. 
Im Zuge dessen wird die Relevanz von Microservices auf der Cloud im Bereich der künstlichen Intelligenz auf einer zukünftigen Produktionsanlage untersucht. 
Die praktische Umsetzung wird durch ein Kubernetes basierendes Cluster mit Edge Devices aus mehreren Knotenpunkten realisiert, welche on-premise und mit einem Zugang zu Cloud Ressourcen aufgesetzt wird. 
Die Realisierung eines solchen Systems erfolgt durch die Containerplattform Kubernetes und mit dem Orchestierungstool Rancher.

\subsection{Anwendungsszenario}
Der Schwerpunkt der zu entwickelnden Anwendung soll ein Dashboard mit Authentifizierungsmechanismus sein. 
Dieser soll Benutzern ermöglichen sich mit ihrem Passwort oder per Gesichtserkennung in ihr Profil einzuloggen. 
Die Daten sollen persistent gespeichert und können bei erneuten Aufruf der Website wieder verwendet werden. 
Die Anwendung muss automatisch auf einem Kubernetes Cluster Deploybar sein. 
Es soll die Charakteristiken und Vorteile einer Microservice Architektur aufweisen. 
Die Anwendung soll Blaupausen Charakter erfüllen, um ähnliche Herausforderungen zu bewerkstelligen. 

\subsection{Stakeholder}
Ein Stakeholder ist eine Person oder Organisiation mit Einfluss auf die Anforderungen eines Systems.
\url{https://www.hanser-elibrary.com/doi/epdf/10.3139/9783446443136.005}
Die Stakeholder sind wie folgt beschrieben:

\subsubsection{Operator}
Die Webanwendung verfügt über eine Benutzeroberfläche die eine Authentifizierung ermöglicht.
Der Anwender kann sich mit seinen Benutzerdaten und Passwort anmelden oder mit Gesichtserkennung. 
Die Daten werden dabei persistent gespeichert und sind bei erneuten Abruf der Anwendung vorhanden. 
Der Benutzer kann nach erfolgreichen Login auf sein Profil zugreifen.

\subsubsection{Kunde}
Die Webanwendung in Form einer Microservice Architektur funktioniert auf dessen Kubernetes fähigen Infrastruktur. 
Der Kunde kann zwischen on-premise und Cloud-Ressourcen für die Verarbeitung der Gesichtserkennung wählen. 
Eine Vorkonfiguration ermöglicht die automatische Zuweisung der gewünschten Hardware Ressourcen.

\subsubsection{Entwickler}
Die Anwendungsentwicklung ist so konzeptioniert, dass zukünftige Entwickler die Strategie der Bereitstellung für eigene Projekte verwenden können.  
Die Webanwendung erfüllt die Charakterisitiken einer Microservice Architektur.

\section{Spezifikation}
\subsection{Funktionale}
\subsection{nicht-Funktionale}
