\chapter{Analyse}

Das vorherige Kapitel beschreibte die nötigen Technologien für die Realisierung und Ausführung einer Anwendung im Microservice Architektur-Stil.
Dieses Kapitel beschäftigt sich nun mit der Analyse für die spätere Konzeptphase.

\section{Modernisierung der Infrastruktur}
Dieser Abschnitt behandelt die aktuellen Bestrebungen der Krones AG hinsichtlich dem Aufbau einer modernen Infrastruktur in Produktionsanlagen. 
Zunächst der Proof of Concept in dem die Umsetzbarkeit eines Kubernetes fähigen Konzept geprüft wurde. 
Danach darauf aufbauende Anwendungsmöglichkeiten im Bereich der Echtzeitkommunikation oder künstlichen Intelligenz. 

\subsection{Proof of Concept}
Die Krones AG entwickelt neue Konzepte, um Produktionsanlagen Standortübergreifend zu modernisieren. 
In einem davon wurde ein Proof of Concept (Poc) mit dem Software Unternehmen SUSE durchgeführt, um die Umsetzbarkeit von Kubernetes Technologien zu evaluieren. 
Diese beinhaltete die Aufrüstung von Produktionsanlage mit Virtual Edge Devices. 
Die über einen Hypervisor Typ 1 zwei Betriebssysteme gleichzeitig ausführen, dem Human Interface (HMI) mit Windows 10 Embedded und Linux SLE-Micro 5.0 Enterprise. 
Die Integration der Geräte ermöglichen die Maschinennahe Nutzung von Echtzeit Informationen, während des Betriebs und dienen als Fundament für den Aufbau einer wartungsfreien Infrastruktur.

\subsubsection{Kubernetes}
Die einzelnen Virtual Edge Devices sollen in zukunft als Knotenpunkte für ein Kubernetes Cluster dienen. 
Dafür wird die für Edge Szenarien entworfene Kubernetes Distribution k3s installiert. 
Die Kubernetes Cluster werden mit dem Orchestrierungstool Rancher verwaltet. 
Diese abstrahiert die Komplexität für den aktiven Betrieb von mehreren Clustern.


\subsection{Microservices}
\subsubsection{Echtzeitkommunikation}
\subsubsection{Künstliche Intelligenz}

\section{Fachkonzept}
Im folgenden Abschnitt werden die Anforderungen und Anwendungsfälle für die spätere Konzeption der Microservice Architektur erhoben.
Dafür wird ein zielorientierter Ansatz gewählt, der Anforderungen aus den Zielen einer Aufgabenstellung herleitet \cite[S.47]{Laplante}. 

\subsection{Anforderungserhebung}
Der vorherige Abschnitt beschreibt die Aufgabenstellungen, die mithilfe von Zielen erreicht werden können. 
Diese sind erforderlich, um die zukünftigen Anforderungen der Software zu erarbeiten und werden in einer Anforderungsmatrix wiedergeben. 

\begin{table}[!htb]
\begin{center}

\begin{tabular}{|p{1cm}|p{6.8cm}|p{6.8cm}|l|l}
    \hline
    \textbf{Nr.} & \textbf{Ziele} & \textbf{Anforderungen} \\
    \hline
    A01 & Auf dem Kubernetes Cluster laufen Anwendungen im Bereich der künstlichen Intelligenz.& Dienste sollen Funktionen der künstlichen Intelligenz ausführen können.\\
    \hline
    A02	& Anwendungen auf einem Kubernetes Cluster haben die Möglichkeit Daten persistent zu speichern.& Dienste sollen über einen Speicher verfügen der Daten persistent speichert.\\
    \hline
\end{tabular}
\caption{Anforderungsmatrix}
\end{center}
\end{table}

\subsection{Anwendungsfälle}