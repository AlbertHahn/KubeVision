\chapter{Lösungskonzept}
Im Fokus des vierten Kapitels steht die Konzeption und Architektur der Microservice-Anwendung.

\section{Design Entscheidungen}

\subsection{Backend-Server}

\subsubsection{Flask}
Das leichtgewichtige Framework Flask wird für die 

\subsubsection{Gunicorn}

\subsection{Frontend}

\subsubsection{HTML}

\subsubsection{JavaScript}

\subsection{Kommunikation}

\subsubsection{SocketIO}

\subsubsection{REST}

\subsection{Gesichtserkennung}

\subsubsection{OpenCV}

\subsection{Datenbank}

\subsubsection{MongoDB}

\section{Architektur}
Dieser Abschnitt befasst sich mit der Architektur der zu entwickelnden Anwendung.
Die Architektur wird in mehrere Bausteinschichten abgebildet.
Der grundlegende Aufbau der Architektur wird seziert und dessen einzelne Komponenten
und Funktionsweisen aufgeklärt.

Dabei geht die Abstraktionsschicht absteigend von der Benutzereingabe bis zum Services.






\subsection{Frontend-Service}


\subsection{Container}
\subsection{Kubernetes}
\subsection{Frontend-Service}
