\chapter{Lösungskonzept}
Im Fokus des vierten Kapitels steht die Konzeption und Architektur der Microservice-Anwendung.

\section{Design Entscheidungen}

Der folgende Abschnitt behandelt die gewählten Technologien für das Lösungskonzept der Microservice-Architektur.

\subsection{Backend}

\subsubsection{Flask}
Für die Entwicklung der Webanwendung in Python wird das Microframework Flask verwendet.
Dieses beinhaltet nur die wesentlichsten Funktionalitäten für die Webentwicklung.
Dafür bietet das Framework eine hohe Flexibilität, da die nötigen Bibliotheken vom Entwickler gewählt werden können \cite{flaskdocu} und
vereinfacht die Erstellung von \acs{api}s \cite[S.11]{restfulpython}.

Die Verwendung von Blueprints ermöglicht die Aufteilung von Endpunkten in views.
Bei Aufrufen eines Endpunkts wird die dazugehörige view ausgegeben \cite{flaskdocu}.
Dadurch kann man die Funktionalität der Webanwendung nach Endpunkten strukturieren und ist ideal für die Ausführung einer Microservice-Architektur.

Weiterhin ermöglichen die Flask-Abhängigkeiten wie Template-Engine Jinja und dem WSIG Werkzeugkasten,
das Rendern von HTML-Templates mit Daten aus der Flask-Anwendung
und dem Bereitstellen einer standartisierten Schnittstelle \ac{wsgi}, welche die Verwendung der meisten Webserver ermöglicht \cite{flaskdocu}.


\subsubsection{Gunicorn}
Gunicorn ist ein \acs{wsgi} \acs{http} server für Unix.
Dieser ist mit den meisten Webframeworks kompatibel.
Das Gunicorn-Modell teilt einen Master-Prozess in mehrere Worker-Prozesse auf.
Der Master-Prozess ist lediglich eine Schleife für die bestehenden Worker-Prozesse
und ist bei Ausfall für den Neustart zuständig.
Die Worker-Prozesse sind für die Verarbeitung von eingehenden Anfragen zuständig.
Diese teilen sich in folgende Worker-Class ein. 
Sync-Workers bearbeiten Anfragen jeweils Einzeln und unterstützten keine persistente Verbindung.
Async-Workers sind basierend auf Greenlets und unterstützten mithilfe von Gevent asynchrone Koroutinen \cite{gunciorndocs}. 

\subsubsection{OpenCV}
OpenCV ist eine Open-Source-Computer-Vision\footnote{Computer-Vision bezeichnet die Transformation von visuellen Daten in eine abgewandelte Form, die zu Beantwortung einer Fragestellung dienen kann.} Bibliothek, die zur Vearbeitung von Bildern verwendet wird \cite{opencvintro}.
Da ein Video nur eine Serie von Bildern ist, können die Techniken der Bildverarbeitung auch hier genutzt werden \cite{ansari_core_2020}.
Die Bibliothek beinhaltet eine Vielzahl an Algorithmen mit Bezug zu Computer-Vision oder Machine-Learning.
Diese untersützten auch die Verwendung von \acp{gpu} die auf den Programmierschnittstellen \ac{cuda} oder \ac{opencl} basieren \cite{opencvpython}.
Die Anwendung wird mit der Python-Version der Bibliothek entwickelt, um ihn in unseren Python-Flask Webanwendung leichter zu integrieren.



\subsection{Frontend}

\subsubsection{\ac{html} und JavaScript}
Die Umsetzung der Benutzerobfläche erfolgt mit der Auszeichnungssprache \acs{html}5 in Kombination
mit der Skriptsprache JavaScript, um die Interaktion des Anwenders zu realisiern.
Für die erleichterte Gestaltung der Website wird das Frontend-\acs{css}-Framework Bootstrap Version 5.0 genutzt.

\subsection{Kommunikation}

\subsubsection{SocketIO}
Die bidirektionale und ereignisbasierte Echtzeitkommunikation zwischen den Diensten wird mithilfe der SocketIO Bibliothek realisiert.
Die Bibliothek unterstützt mehrere Programmiersprachen für Server und Client Implementierung, welche von der Community gewartet werden.
Eine Kommunikation zwischen Server und Client erfolgt über WebSockets, wenn dies nicht möglich ist, wird auf die Ressourcen intensivere \cite{httpwebsocket} Alternative \acs{http}-long-polling zurückgegriffen \cite{socketio}.
Für die Kommunikation der Webanwendung wird die Server Implementierung von Python-Socketio genutzt \cite{python-socketio}.
Die Implementierung der Anwenderlogik erfolgt über die JavaScript SocketIO Bibliothek, diese sind Plattformunabhängig.


\subsubsection{\ac{rest}}
\acs{rest} ist ein Architektur-Stil für verteilte Systeme.
Dieser ist ein Ressourcen basierende Architektur, die zugänglich über übliche 

\subsection{Datenbank}

\subsubsection{MongoDB}

\section{Architektur}
Dieser Abschnitt befasst sich mit der Architektur der zu entwickelnden Anwendung.
Die Architektur wird in mehrere Bausteinschichten abgebildet.
Der grundlegende Aufbau der Architektur wird seziert und dessen einzelne Komponenten
und Funktionsweisen aufgeklärt.

Dabei geht die Abstraktionsschicht absteigend von der Benutzereingabe bis zum Services.






\subsection{Frontend-Service}


\subsection{Container}
\subsection{Kubernetes}
\subsection{Frontend-Service}
