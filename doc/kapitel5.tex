\chapter{Implementierung der Architektur}

Das folgende Kapitel beschreibt die Vorgehensweisen der Implementierung.
Angefangen mit dem Aufbau der einzelnen Schichten der Anwendung.


\section{Konfiguration und Einrichtung}

In diesem Abschnitt geht es um die Einrichtung der Kubernetes Infrastruktur.
Zuerst die Einrichtung der einzelnen virtuellen privaten Server in Vultr, die als Knotenpunkte in unserem Kubernetes Cluster funktonieren.
Danach die Installation der Infrastruktur mit k3s.
Als nächstes kommt die Konfiguration einer Domain zum Einsatz, um die Microservices zu verwenden.
Abschließend erfolgt die Bereitstellung von Zertifikaten für unsere Domain.

\subsection{Virtueller Privater Server}

Durch die Einschränkungen, beschrieben in Abschnitt \ref{Einschraenkungen},
wird für die Installation der Kubernetes Plattform virtuelle private Server (VPS) verwendet.
Ein VPS ist eine virtuelle Maschine, die von Drittanbietern wie Internet-Hosting-Diensten, als Dienst verkauft wird.
Dies ermöglicht das Mieten von Hardware für einen monatlichen Geldbeitrag.
Die Server dienen als Knotenpunkte für die spätere Kubernetes Installation.
Es werden ingesamt drei VPS-Instanzen gemietet auf denen das Betriebssystem installiert wird.

\subsubsection{Betriebssystem}
Im Rahmen des PoC mit dem Unternehmen SUSE wurde das 
Betriebssystem SLE-Micro Enterprise 5.1 bereitgestellt und auf den Serverinstanzen installiert.
Das Betriebssystem ist für den Einsatz von IoT und Edge-Device Szenarien bestimmt.
Dieses arbeitet mit transactional-updates, welche Updates erst aktivieren, wenn das Betriebssystem neugestartet wurde. 
Erfolgt das Update nicht wird ein rollback des vorherigen Versionszustand hergestellt.
Für die spätere Installation von k3s wird eine fehlende Abhängigkeit auf dem Betriebssystem benötigt \ref{lst:apparmor}.

\begin{lstlisting}[caption={apparmor-parser command},captionpos=b,label={lst:apparmor},language=bash]
transactional-update pkg install apparmor-parser
\end{lstlisting}

\subsubsection{Domain}
Der Zugang zur Webanwendung wird mithilfe einer öffentlichen Domain ermöglicht.
Der DNS-Eintrag einer Domain ist für die Adressierung zuständig.
Durch die Veränderung des A-Records leiten wir alle Anfragen der Domain auf eine IPv4-Adresse um \cite{LearningCoreDNS}.
Die IPv4 Adresse ist in diesem Fall, der Cluster Master unserer späteren k3s-Installation.

\subsection{Kubernetes Installation}
Dieser Abschnitt behandelt die Einrichtung und Installation der Kubernetes Distribution Lightweight Kubernetes und der Orchestrierungsplattform Rancher.


\subsubsection{Lightweight Kubernetes}\label{k3screate}
Für die Installation von k3s auf den Server wurde ein Shell Skript entwickelt, 
dass mit den nötigen Befehlen aus der Dokumentation geschrieben wurde.
\begin{lstlisting}[caption={Ausschnitt des installk3s.sh Skripts},captionpos=b,label={lst:k3sintall},language=bash]
    curl -sfL https://get.k3s.io | INSTALL_K3S_EXEC="server" K3S_CLUSTER_INIT=1 sh -
    TOKEN=$( cat /var/lib/rancher/k3s/server/node-token )
    
    USERNAME=root
    SCRIPT="curl -sfL https://get.k3s.io | INSTALL_K3S_EXEC="server" K3S_TOKEN=$TOKEN K3S_URL=https://$ip4:6443 sh - "
    for HOSTNAME in ${HOSTS[@]} ; do
        ssh -o StrictHostKeyChecking=no -l ${USERNAME} ${HOSTNAME} "hostnamectl set-hostname ${HOSTNAMES[$COUNTER]}; ${SCRIPT}"  
        echo "HOSTNAME CHANGED: ${HOSTNAMES[$COUNTER]}"
        ((COUNTER++))
    done
\end{lstlisting}

Das Skript holt mittels \textit{curl} Aufruf das Installationsskript für k3s und installiert es mit den vorgegebenen Initialisierungsparametern. 
Der Parameter \textit{INSTALL\_K3S\_EXEC} bestimmt die Aufgabe der Node.
\textit{K3S\_CLUSTER\_INIT} initialisiert die Node als neuen Cluster-Master.
Danach wird ein Token im \textit{../k3s/server} Verzeichnis angelegt, dieser ist für die Verknüpfung der andern Knoten nötig und wird als Variable gespeichert.
In einer Schleife werden die vom Skript vorher abgefragten IP-Adressen und Hostsystemnamen des Benutzer verarbeitet.
Über das Secure Shell (SSH) Netzwerkprotokoll verbindet sich das ausführende System mit den restlichen Knotenpunkten
und installiert mithilfe der Variablen \textit{TOKEN} und \textit{SCRIPT} k3s.

\subsubsection{Rancher}
Im Rahmen des PoCs wurde bereits ein Rancher Server zur Verwaltung von mehreren downstream Cluster erstellt.
Der im vorherigen Abschnitt \ref{k3screate} eingerichtete k3s Cluster wird mit dem Rancher Server verbunden.
Über die Rancher Benutzeroberfläche lässt sich das Kubernetes Cluster mithilfe des \textit{Add Cluster} Buttons hinzufügen.
Die weiteren Schritte ermöglichen, dass benennen des Cluster und den notwendigen Befehl zum verknüpfen des gewünschten Clusters.

\subsection{Microservice}
Dieser Abschnitt behandelt den Aufbau der Anwendung im Microservice-Stil.
Der Aufbau der Anwendung zur Gesichtserkennung in ist in mehrere Services aufgeteilt.

\subsection{Frontend-Service}
\subsection{Authentifizierungs-Service}
\subsection{Backend-Service}


%alles ausschreiben??
%\section{Frameworks und Bibliotheken für Microservices}
%\subsection{Flask}
%\subsection{Gunicorn}
%\subsection{SocketIO}
%\subsection{OpenCV}
%\subsection{MongoDB}

\section{Gesichtserkennung}
\subsection{Alignment}
\subsection{Training}
\subsection{Model}

\section{Containerisierung}
\subsection{Volumes}
\subsection{Netzwerk}
\subsection{Docker-Compose}
\subsection{DockerHub}

\section{Orchestrierung}
\subsection{SSL-Verschlüsselung}
\subsection{Deployment}
\subsection{Ingress}
\subsubsection{Nginx-Ingress}
\subsection{Loadbalancer}
\subsection{Taints and Tolerations}
\subsection{Node Affinity}
\subsection{Helm}


\section{Testen der Implementeriung}
\subsection{Service Kommunikation}
\subsection{Loadbalancing}
\subsection{Gesichtserkennung}